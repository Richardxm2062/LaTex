\documentclass{article}    

%文章相关
\usepackage[UTF8, heading = false, scheme = plain]{ctex}    %解决中文字体,不改变排版
\usepackage{geometry}                                       %调整页边距等
\usepackage{indentfirst}                                    %首行缩进
\usepackage[dvipsnames,svgnames]{xcolor}                    %颜色包:\color{}



%图片宏包
\usepackage{graphicx}                                       %插入图片:\includegraphics{myimage.png}
\usepackage{float}
\usepackage{caption}
\usepackage{subcaption}

%数学相关
\usepackage{amsmath,amsthm}                                 %amsmath包应该在前


%实用的内容说明包
\usepackage{hyperref}                                       %超链接插入包:\href{url}{name}
\usepackage{multirow}                                       %插入表格用到的宏包:\begin{tabular}{|c|c|},&分列,\\分行,\hline插入横线(第二个参数是列数)
\usepackage{listings}                                       %插入代码等:\begin{lstlisting}[breaklines=true,backgroundcolor=\color{lightgray},title=]
\usepackage{verbatim}                                       %使用 comment 环境进行注释
\usepackage{extarrows}                                      %使用长等号\xlongequal

%物理包
\usepackage{physics}
\usepackage{xparse}                                         %physics需求的包

% \eval{}        根据括号内容的大小在右边添加合适的竖线A|          |         \dv[n]{f}{x}          n阶微分算符 derivative
% \abs{}         添加绝对值|A|                                |         \pdv[n]{f}{x}         n阶偏微分算符 partialderivative
% \norm{}        模||A||                                     |          \bra \ket \braket    左、右、内积
% \comm{}{}      对易子[A,B]                                  |          \op{}{}              外积(密度算符) outerproduct
% \anticomm{}{}  反对易子{A,B}                                |          \ev{<力学量>}{<态>}    期望 expectationvalue
% \vb{}          向量加粗字体                                  |          \mel{}{}{}            矩阵元 matrixelement
% \va{}          向量                                        |           \mqty                生成矩阵,可跟{} () [] || &分列 \\分行 第一个表示外面无框
% \vu{}          单位向量                                     |          \qty                  合适大小的{}、()等,例如 \qty(x)
% \vdot          点乘                                        |
% \cross         叉乘                                        |
% \grad          梯度 gradient                               |
% \div           散度 divergenc                              |
% \curl          旋度                                         |
% \laplacian     拉普拉斯算符                                  |

%文本底纹实现
\usepackage{lipsum}                                         %该宏包是通过 \lipsum 命令生成一段本文,正式使用时不需要引用该宏包
\usepackage[strict]{changepage}                             %提供一个 adjustwidth 环境
\usepackage{framed}                                         %实现方框效果
\usepackage{newtxtext}
\usepackage{tcolorbox}                                      %文本底纹包,放在xcolor包后
                               
% environment derived from framed.sty: see leftbar environment definition
\definecolor{formalshade}{rgb}{0.95,0.95,1}% 文本框颜色
% ------------------******-------------------
% 注意行末需要把空格注释掉,不然画出来的方框会有空白竖线
\newenvironment{formal}{%
\def\FrameCommand{%
\hspace{-1em}%                                              %框的整体缩进
{\color{Green}\vrule width 0.3em}%                          %竖线颜色以及宽度
\colorbox{greenshade}%                                      %底纹颜色
}%
\MakeFramed{\advance\hsize-\width\FrameRestore}%
\noindent\hspace{-2em}%                                     %禁止第一段文本缩进
\begin{adjustwidth}{}{2em}%                                 %控制右边距        
\vspace{1.5em}%                                             %控制上边距
}
{%
\vspace{1.5em}\end{adjustwidth}\endMakeFramed%              %控制底边距
}%

\definecolor{greenshade}{rgb}{0.90,0.99,0.91}               %绿色文本框,竖线颜色设为 Green
\definecolor{redshade}{rgb}{1.00,0.90,0.90}                 %红色文本框,竖线颜色设为 LightCoral
\definecolor{brownshade}{rgb}{0.99,0.97,0.93}               %莫兰迪棕色,竖线颜色设为 BurlyWood


%使用Mathematica进行计算
\usepackage{latexalpha2}
%/usr/local/texlive/texmf-local/tex/latex/local/latexalpha2
%直接使用wolfram代码
% \wolfram[<format>]{<code>}    \wolframgraphics[<format>]{<code>}{<filename>} 
%具体使用方法查看 latexalpha2.pdf



%预设
\geometry{a4paper,left=5em,right=5em,bottom=5em,top=5em}    %设置为a4paper最好,点击pacakge geometry 查看文档
\setlength{\parindent}{2em}                                 %2em(注意不支持rem)代表每一段的首行缩进两个字符,某一行不缩进时使用 \noindent

\hypersetup{hidelinks,colorlinks=true,
linkcolor=black,urlcolor=blue}                              %对hyperref 包进行预设

\newenvironment{slt}{\proof[\indent \bf 解 ]}{
\renewcommand{\qedsymbol}{}\endproof}                       %提供解环境
\newtheorem{thm}{定理}[section]                              %定义一个新的环境 thm, 命名为定理,以 节 开始编号
\newtheorem{lemma}{引理}[section]                            %定义新环境 lemma,命名为引理,以 节 开始编号
\newtheorem{corollary}{推论}                                 %定义新环境 corollary,命名为推理,无编号
\renewcommand{\proofname}{ \qquad \bf 证明}                  %更改proof为中文证明,proof环境默认存在

%一些def
\def\thmindent{\setlength{\parindent}{5em}}                  %\thmindent
\def\pfindent{\setlength{\parindent}{5.5em}}                 %\pfindent
\def\clindent{\setlength{\parindent}{4em}}                   %clindent
\def\sdr{Schr\"{o}dinger}                                    %薛定谔名字
\def\intff{\int_{-\infty}^{+\infty}}                         %积分为(-\infty,+\infty)的积分
\def\lra{\Longrightarrow}                                    %长右键头\lra
\def\llra{\Longleftrightarrow}                               %等价箭头\llra

\def\psii#1{\psi_{#1}}                                       %常用的\psi下标
\def\psiii#1#2{\psi_{#1} (#2)}                               %常用的\psi下标和括号
\def\psiiii#1#2#3{\psi_{#1}^{#2} (#3)}                       %常用的\psi下标、上标和括号
\def\pe#1#2{E_{#1}^{(#2)}}                                   %能量修正
\def\pp#1#2{\psi_{#1}^{(#2)}}                                %波函数修正
\def\ua{a_{+}}                                               %升算符
\def\da{a_{-}}                                               %降算符


%其他备注
\begin{comment}
        求和指标上下方添加        \sum\limits_{}^{}
        恒等于                  \equiv
        远大于                  \gg
        远小于                  \ll


\end{comment}




%作业包(需要的时候再解除注释)
%\usepackage{iidef}
%建议使用自己的slt解环境
\begin{comment}

    package iidef:
        指定学校名      \thecourseinstitute{}      
        指定课程名      \thecoursename{} 
	    指定学期        \theterm{}
        作业名         \hwname{}
        生成作业标题    \courseheader           放在document环境内 不需要再maketile
        名字           \name                   放在document环境内
        自动编号环境    \begin{enumerate}       
        题号           \item               自动编号 
        证明           \begin{proof}           证明环境由amsthm包提供
        求解           \begin{solution}       
        方程           \begin{equation}        
        方程编号        \labe{eq:[number]}      为方程设置编号  
        引用房产        \eqref{eq:[number]}     引用方程
        行内方程        $...$
        多行公式        \begin{align}           
                            ... & = ... \\ 
                                & = ...
                       \begin{array}{lcl}      
                            ... & = & ... \\ 
                            ... & = & ...
        方程组          \begin{cases}
                            ... & \mbox{if} x \mbox{is even} \\ 带假设

\end{comment}


