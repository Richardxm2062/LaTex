\documentclass{article}    

\usepackage[UTF8, heading = false, scheme = plain]{ctex}    %解决中文字体,不改变排版
\usepackage{geometry}   %调整页边距等
\usepackage{indentfirst}    %首行缩进

\usepackage{graphicx}   %插入图片:\includegraphics{myimage.png}
\usepackage{float}
\usepackage{caption}
\usepackage{subcaption}

\usepackage{hyperref}   %超链接插入包:\href{url}{name}
\usepackage{multirow}   % 插入表格用到的宏包:\begin{tabular}{|c|c|}
\usepackage{xcolor}  %颜色包:\color{}
\usepackage{listings}   %插入代码等:\begin{lstlisting}[breaklines=true,backgroundcolor=\color{lightgray},title=]
\usepackage{verbatim}   %使用 comment 环境进行注释
\usepackage{amsmath,amsthm}  %amsmath包应该在前
\usepackage{braket} %狄拉克符号系统 \bra{} \ket{}




\geometry{a4paper,left=5em,right=5em,bottom=5em,top=5em}    %设置为a4paper最好,点击pacakge geometry 查看文档
\setlength{\parindent}{2em}    %2em(注意不支持rem)代表每一段的首行缩进两个字符,某一行不缩进时使用 \noindent
\hypersetup{hidelinks,colorlinks=true,linkcolor=black,urlcolor=blue}   %对hyperref 包进行预设
\newtheorem{thm}{定理}[section]     %定义一个新的环境 thm, 命名为定理,以 节 开始编号,数学论文中常用
\renewcommand{\proofname}{ \qquad \bf 证明} %更改proof为中文证明
\newenvironment{solution}{\proof[\indent \bf 解]}{\renewcommand{\qedsymbol}{}\endproof} %提供解环境
\newtheorem{lemma}{引理}[section]
\newtheorem{corollary}{推论}
\def\pfindent{\setlength{\parindent}{5.5em}}
\def\thmindent{\setlength{\parindent}{5em}}
\def\clindent{\setlength{\parindent}{4em}}