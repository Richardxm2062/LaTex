
%文章相关
\usepackage[UTF8, heading = false, scheme = plain]{ctex}    %解决中文字体,不改变排版
\usepackage{geometry}                                       %调整页边距等
\usepackage{indentfirst}                                    %首行缩进
\usepackage[dvipsnames,svgnames]{xcolor}                    %颜色包:\color{}
\usepackage{enumitem}                                       %控制item的前置符号
\usepackage{adjustbox}                                      %控制文字大小和添加底纹


%图片宏包
\usepackage{graphicx}                                       %插入图片:\includegraphics{myimage.png}
\usepackage{float}
\usepackage{caption}
\usepackage{subcaption}

%数学相关
\usepackage{amsmath,amsthm}                                 %amsmath包应该在前
\usepackage{extarrows}                                      %使用长等号A \xlongequal{\quad\quad}B
\usepackage{cases}                                          %\begin{numcases}可以为方程编号
\usepackage{amssymb}                                        %\checkmark 打勾

%实用的内容说明包
\usepackage{hyperref}                                       %超链接插入包:\href{url}{name}
\usepackage{multirow}                                       %插入表格用到的宏包:\begin{tabular}{|c|c|},&分列,\\分行,\hline插入横线(第二个参数是列数)
\usepackage{listings}                                       %插入代码等:\begin{lstlisting}[breaklines=true,backgroundcolor=\color{lightgray},title=]
\usepackage{verbatim}                                       %使用 comment 环境进行注释


%物理包
\usepackage{physics}
\usepackage{xparse}                                         %physics需求的包
\usepackage{mhchem}                                         %输入原子核的表式\ce{^238_92U}

% \eval{}        根据括号内容的大小在右边添加合适的竖线A|          |         \dv[n]{f}{x}          n阶微分算符 derivative
% \abs{}         添加绝对值|A|                                |         \pdv[n]{f}{x}         n阶偏微分算符 partialderivative
% \norm{}        模||A||                                     |          \bra \ket \braket    左、右、内积
% \comm{}{}      对易子[A,B]                                  |          \op{}{}              外积(密度算符) outerproduct
% \anticomm{}{}  反对易子{A,B}                                |          \ev{<力学量>}{<态>}    期望 expectationvalue
% \vb{}          向量加粗字体                                  |          \mel{}{}{}           矩阵元 matrixelement
% \va{}          向量                                        |          \mqty                 生成矩阵,可跟{} () [] || &分列 \\分行 第一个表示外面无框
% \vu{}          单位向量                                     |         \qty                  合适大小的{}、()等,例如 \qty(x)
% \vdot          点乘                                        |
% \cross         叉乘                                        |
% \grad          梯度 gradient                               |
% \div           散度 divergenc                              |
% \curl          旋度                                        |
% \laplacian     拉普拉斯算符                                 |

%文本底纹实现
\usepackage{lipsum}                                         %该宏包是通过 \lipsum 命令生成一段本文,正式使用时不需要引用该宏包
\usepackage[strict]{changepage}                             %提供一个 adjustwidth 环境
\usepackage{framed}                                         %实现方框效果
%\usepackage{newtxtext}                                     %在macos会报错的一个包注释掉不影响使用
\usepackage{tcolorbox}                                      %文本底纹包,放在xcolor包后
                               
% environment derived from framed.sty: see leftbar environment definition
\definecolor{formalshade}{rgb}{0.95,0.95,1}% 文本框颜色
% ------------------******-------------------
% 注意行末需要把空格注释掉,不然画出来的方框会有空白竖线
\newenvironment{formal}{%
\def\FrameCommand{%
\hspace{-1em}%                                              %框的整体缩进
{\color{Green}\vrule width 0.3em}%                          %竖线颜色以及宽度
\colorbox{greenshade}%                                      %底纹颜色
}%
\MakeFramed{\advance\hsize-\width\FrameRestore}%
\noindent\hspace{-2em}%                                     %禁止第一段文本缩进
\begin{adjustwidth}{}{2em}%                                 %控制右边距        
\vspace{1.5em}%                                             %控制上边距
}
{%
\vspace{1.5em}\end{adjustwidth}\endMakeFramed%              %控制底边距
}%

\definecolor{greenshade}{rgb}{0.90,0.99,0.91}               %绿色文本框,竖线颜色设为 Green
\definecolor{redshade}{rgb}{1.00,0.90,0.90}                 %红色文本框,竖线颜色设为 LightCoral
\definecolor{brownshade}{rgb}{0.99,0.97,0.93}               %莫兰迪棕色,竖线颜色设为 BurlyWood


%使用Mathematica进行计算
%\usepackage{latexalpha2}
%/usr/local/texlive/texmf-local/tex/latex/local/latexalpha2
%直接使用wolfram代码
% \wolfram[<format>]{<code>}    \wolframgraphics[<format>]{<code>}{<filename>} 
%具体使用方法查看 latexalpha2.pdf


%预设
\geometry{a4paper,left=5em,right=5em,bottom=5em,top=5em}    %设置为a4paper最好,点击pacakge geometry 查看文档
\setlength{\parindent}{2em}                                 %2em(注意不支持rem)代表每一段的首行缩进两个字符,某一行不缩进时使用 \noindent

\hypersetup{hidelinks,colorlinks=true,
linkcolor=black,urlcolor=blue}                              %对hyperref 包进行预设

\newenvironment{slt}{\proof[\indent \bf 解 ]}{
\renewcommand{\qedsymbol}{}\endproof}                       %提供解环境
\newtheorem{thm}{定理}[section]                              %定义一个新的环境 thm, 命名为定理,以 节 开始编号
\newtheorem*{thm*}{定理}                                     %定义一个新的环境 thm, 命名为定理,无编号
\newtheorem{lemma}{引理}[section]                            %定义新环境 lemma,命名为引理,以 节 开始编号
\newtheorem*{lemma*}{引理}                                   %定义新环境 lemma,命名为引理,无编号
\newtheorem{corollary}{推论}                                 %定义新环境 corollary,命名为推理,有编号
\newtheorem*{corollary*}{推论}                               %定义新环境 corollary*,命名为推理,无编号
\renewcommand{\proofname}{ \qquad \bf 证明}                  %更改proof为中文证明,proof环境默认存在

%一些def
\def\thmindent{\setlength{\parindent}{5em}}                  %\thmindent
\def\pfindent{\setlength{\parindent}{5.5em}}                 %\pfindent
\def\clindent{\setlength{\parindent}{4em}}                   %clindent
\def\sdr{Schr\"{o}dinger}                                    %薛定谔名字
\def\intff{\int_{-\infty}^{+\infty}}                         %积分为(-\infty,+\infty)的积分
\def\ra{\rightarrow}                                         %右键头\ra
\def\lra{\Longrightarrow}                                    %长(双)右键头\lra
\def\lla{\Longleftarrow}                                     %长(双)左键头\lla
\def\llra{\Longleftrightarrow}                               %等价箭头\llra
\def\xlra{\xlongrightarrow{\quad\quad}}                      %超长右箭头
\def\xlla{\xlongleftarrow{\quad\quad}}                       %超长左箭头
\def\xlla{\xlongleftrightarrow{\quad\quad}}                  %超长等价箭头

\def\psii#1{\psi_{#1}}                                       %常用的\psi下标
\def\psiii#1#2{\psi_{#1} (#2)}                               %常用的\psi下标和括号
\def\psiiii#1#2#3{\psi_{#1}^{#2} (#3)}                       %常用的\psi下标、上标和括号
\def\pe#1#2{E_{#1}^{(#2)}}                                   %能量修正{下标}{上标}
\def\pp#1#2{\psi_{#1}^{(#2)}}                                %波函数修正
\def\ua{a_{+}}                                               %升算符
\def\da{a_{-}}                                               %降算符

\def\nuc#1#2#3{\ce{^{#1}_{#2}{#3}}}                          %原子核的表示形式


%其他备注
\begin{comment}
        求和指标上下方添加        \sum\limits_{}^{}
        恒等于                  \equiv
        远大于                  \gg
        远小于                  \ll
        花括号                  \left\{ \right\}   (建议使用\qty)
        弧度                    37^{\circ}


\end{comment}


%作业包(需要的时候再解除注释)
%\usepackage{iidef}
%建议使用自己的slt解环境
\begin{comment}

    package iidef:
        指定学校名      \thecourseinstitute{}      
        指定课程名      \thecoursename{} 
	    指定学期        \theterm{}
        作业名         \hwname{}
        生成作业标题    \courseheader           放在document环境内 不需要再maketile
        名字           \name                   放在document环境内
        自动编号环境    \begin{enumerate}       [label = (\alph*{})] [label = \arabic*{}.]
        题号           \item                   自动编号,\item[]则不带符号
        证明           \begin{proof}           证明环境由amsthm包提供
        求解           \begin{solution}       
        方程           \begin{equation}        
        方程编号        \labe{eq:[number]}      为方程设置编号  
        引用方程        \eqref{eq:[number]}     引用方程
        行内方程        $...$
        
        多行公式        \begin{align} \begin{align*}则不会编号  
                            ... & = ... \\ 
                                & = ...
                       \begin{array}{lcl}      
                            ... & = & ... \\ 
                            ... & = & ...
        
        方程组          $$
                        \begin{cases}
                       ... & \mbox{if} x \mbox{is even} \\ 带假设
                       $$
                       
                       带编号
                       \begin{numcases}{}
                            \label{1}   \\
                            \label{2}   
                       \end{numcases}
        
        文字大小和底纹
                       \vspace{-1em}
                       \begin{adjustbox}{minipage=0.91\linewidth, bgcolor=gray!20, padding=1em}
                       \small % 将字号变小为 small
                        text
                       \end{adjustbox}
                       \vspace{-1em}

        居中            不要使用$$...$$,会对齐失效 使用$...$即可
                        \begin{center}
            
                        \end{center}

        左对齐           
                        \begin{flushleft}
            
                        \end{flushleft}

        双水平图         \begin{minipage}{0.45\textwidth}
                        \includegraphics[width=\textwidth,keepaspectratio]{./pictures/.png}
                        \end{minipage}
                        \hfill
                        \begin{minipage}{0.45\textwidth}
                            \begin{enumerate}[label = (\arabic*)]
                                \item 
                                \item 
                                \item 
                            \end{enumerate}
                        \end{minipage}

    
\end{comment}



\title{高一下期末题解}
\author{richardxm}

\begin{document}

    \maketitle
    \tableofcontents
    \newpage


    \section{选择题部分解}

        \begin{enumerate}
            \item B 
            \item A     \newline
            A选项的\textbf{限定词}是自由落体运动,因此无论如何只受到重力影响,无其他外力损耗该系统因此是机械能守恒的   \newline 
            B选项可以直接从数学定义式理解$ W = \vec{F} \vdot \vec{S} $ \quad $ E_{k} = \frac{1}{2} m v^{2} $,显然动能大小仅仅取决于速率,而速率和功正负或者说与$\vec{F}$,$\vec{S}$无直接关系     \newline
            C选项滑动摩擦力的定义是:具有接触且有相对运动的粗糙物体之间产生的力,而做正功还是做负功必须要明确到底是对哪一个物体,因此并非一定做负功
            \item A     \newline
            电荷不会凭空产生,以正负电荷成对出现
            \item C     \newline 
            侧面感应出负电荷,且越靠近中心电荷密度越大,相应的电场线越密集
            \item B     
            $$
            F_{n} = \frac{m v^{2}}{r} = G\frac{mM}{r^{2}} = m a_{n} \lra v^{2} = \frac{GM}{r}
            $$
            A选项同一个P点受到的万有引力一样大提供全部向心力,因此加速度一样大   \newline
            B选项调相轨道整体大于停泊轨道,在经过P点时需要万有引力大于所需的向心力,因此减速即可  \newline
            C选项,开普勒第三定律$ \frac{a^{3}}{T^{2}} = k $($k$是一个常数仅仅与中心天体有关),调相轨道的半长轴更大因此周期更大   \newline
            D选项,在运动过程中机械能守恒,不妨考虑P点引力势能一样大,调相轨道的速度更大因此动能更大,所以在此轨道上机械能更大
            \item D     \newline
            题干中的\textbf{分别}理解到位就好做了,在两种外电阻的情况下唯一不变的就是总电压,通过热量求出两种电路的电流关系再根据电压列等式可得内电阻
            \item C  \newline
            A选项,轨迹在后半段收到向下的电场力,电场方向趋势为右上,因此带\textbf{负电荷}     \newline
            B选项,后半段电场线变疏受力变小      \newline
            C选项,电荷为负电荷,显然电荷有一定初速度,且电场力持续做负功,速度一直减小     \newline 
            D选项,机械能守恒,动能在持续减小,电势能在增大
            \item C \newline 
            选项A,显然带正电    \newline 
            选项B,电场可以分解为两个垂直方向的电场$E_{BA}$与$E_{CB}$,$U_{BA} = E_{BA} L_{BA} = U_{CB} = E_{CB} L_{CB}$,带入数值即可求的$E_{BA} = \frac{1}{2}$与$E_{CB} = \frac{2}{3}$,合成后可得
            原电场强度为$ E = \frac{5}{6} V/m $     \newline 
            选项C,机械能守恒    \newline 
            选项D存疑
            \item B     \newline 
            选项A,开关S断开两条路并联,由于电容特性分得全部电压,因此可以认为$C_{1}$左侧为18V,$a$点为0V,$R_{1}$左侧为18V,$b$点为18V,$C_{2}$右侧为0V,因此$U_{ab}$为18V     \newline 
            选项B,开关断开时两个电容器均在充电,开关闭合后电容器开始放电     \newline 
            选项C,$ Q = CU $    \newline 
            选项D,闭合后放电到均不带电荷
            \item D     \newline  
            选项A,$F_{all} =  \frac{ \left(\sqrt{3} + 3 \right)  G m^{2} }{L^{2}}$    \newline 
            选项B,$ \frac{m v^{2}}{\frac{\sqrt{3}}{3} L} = F_{all} $, 算出$mv^{2}$后再算动能乘以3得到三星总动能为$ E_{k} = \frac{\sqrt{3} \left( 3 + \sqrt{3} \right) G m^{2} }{2L} $   \newline 
            选项C,$ m \omega^{2} \frac{\sqrt{3}}{3} L = F_{all} = \frac{ \left(\sqrt{3} + 3 \right)  G m^{2} }{L^{2}} \lra w^{2} \propto m $,质量变为两倍,角速度变成$\sqrt{2}$倍    \newline 
            选项D,周期计算$ T = \frac{2\pi}{\omega} $,由选项C计算出$ \omega^{2} \propto \frac{1}{L^{3}} $,所以$L$变为两倍则$\omega^{2}$变为原来的$\frac{1}{8}$,显然$ T^{2} \propto \frac{1}{\omega^{2}} $,$T^{2}$变为原来的
            8倍,开根号得到$2\sqrt{2}$

        \end{enumerate}


    \section{多选题解}
        \begin{enumerate}
            \item CD \newline 
            选项A显然错误       \newline 
            选项B,考虑整个过程中的摩擦力做功大小,$AB$路线一致因此机械能减少一样多,$C$走的路线最长,机械能损失最多.由此$B$在底端全部机械能为动能且初始机械能大于$A$,所以$B$的动能最大  \newline 
            选项C与D可由B的推理过程得出     \newline 
            \item AC \newline 
            选项A,电势能的变化存在拐点,在电势能达到最大值意味着电场力不做功,但是电子由题意并未停止过运动,因此在此处电场强度为0,那么$AB$为异种电荷,且初期收到向左的电场力后期为向右的电场力可推断$A$带负电荷,$B$带正电
            荷,随着距离变大,负电荷的作用越发明显因此它的电荷应该更大        \newline 
            选项B,电子受力向左因此电场线沿$x$轴正方向       \newline 
            选项C,$ k\frac{q_{1}q}{ \left( x_{0}+ x_{2} \right)^{2}} = k\frac{q_{2}q}{x_{2}^{2}}  \lra \frac{q_{1}}{q_{2}} = \frac{\left( x_{0} + x_{2} \right)^{2}}{x_{2}^{2}} $
            \item ABD       \newline
            $ R_{2} $ 与$ R_{3} $ 并联 再和$ R_{1} $ 串联,滑动变阻器向上滑动接入电路的电阻值变大,因此$U_{1}$变小,电路总电阻值变大$I_{1}$减小,$ R_{2} $并联电路整体分压变大此分支电流增大,因此另一分支电流减,
            所以$I_{3}$变小     \newline 
            选项D,$ \frac{\Delta U_{3}}{\Delta I_{1}}$ ,显然电路总电压从未变过,因此有$ \abs{\Delta U_{3}} = \abs{\Delta U_{1}} + \abs{\Delta U_{r}}$,因此前面的比值就是定值电阻$ R_{1} + r $
            \item CD    \newline 
            $t=0$时刻发射的粒子正好从$B$发射出去时所经过的时间由水平位移决定$ t = \frac{2d}{v_{0}} $,恰好为两个周期,竖直方向上反复进行匀加速和匀减速运动,在时间$ t = 2T $里竖直位移恰好是$d$.$ E d = \varphi _{0} $,4
            端匀加减速运动,每段时间为$\frac{1}{2} T$,$ \frac{1}{2} a \left(  \frac{1}{2} T \right)^{2} \vdot 4 = d $,得到$ \frac{1}{2} a T^{2}  = d $,带入$T = \frac{d}{v_{0}} \quad a = \frac{Eq}{m} = \frac{\varphi_{0} q}{md} $,
            注意比荷的定义是$\frac{q}{m}$,结果为$\frac{2v_{0}^{2}}{varphi_{0}}$     \newline 
            选项C,$\frac{1}{4}T$射入意味着电场力做功在$2T$时间里为0,电势能不变     \newline
            选项D,意味着竖直方向上速度不会超过$v_{0}$,电场力能做最多正功的初射时间就是$t=0$,而此时出射速度在竖直方向上为0,此选项正确 
            \item ACD   \newline
            选项A,到达$b$点时减少的机械能仅仅物体$B$收到的摩擦力做功,注意轮轴大小不一样,因此物体$B$移动的距离为绳子变长距离的一半,$ W = (10 - 8) \vdot \frac{1}{2} Mg \cos{\frac{\pi}{6}} \vdot \frac{\sqrt{3}}{3} = 10 J$     \newline 
            选项B,它们的速度比由于轮的存在,沿绳速度满足:圆环沿绳的速度为物体速度的两倍,所以$ V_{A} = V_{B} \vdot 2 \vdot \frac{10}{6}$,所以比值应该为$ 10:3 $   \newline 
            选项C,下降15米则圆环重力势能减少$90J$,此时绳子长$L = 17 m$,物体沿绳上升距离为$(17-8) \vdot \frac{1}{2} = 4.5  m$,获得重力势能$ 45J $,机械能损失$ Mg\cos{\frac{\pi}{6}} \vdot \frac{\sqrt{3}}{3} \vdot  4.5  = 45J $       \newline 
            选项D,显然正确13123123444
        \end{enumerate}

\end{document}