\input{../premble.premble.tex}

\title{高一下期末题解}
\author{}

\begin{document}
    \maketitle
    \tableofcontents
    \newpage


    \section{选择题部分解}

        \begin{enumerate}
            \item B 
            \item A     \newline
            A选项的\textbf{限定词}是自由落体运动,因此无论如何只受到重力影响,无其他外力损耗该系统因此是机械能守恒的   \newline 
            B选项可以直接从数学定义式理解$ W = \vec{F} \vdot \vec{S} $ \quad $ E_{k} = \frac{1}{2} m v^{2} $,显然动能大小仅仅取决于速率,而速率和功正负或者说与$\vec{F}$,$\vec{S}$无直接关系     \newline
            C选项滑动摩擦力的定义是:具有接触且有相对运动的粗糙物体之间产生的力,而做正功还是做负功必须要明确到底是对哪一个物体,因此并非一定做负功
            \item A     \newline
            电荷不会凭空产生,以正负电荷成对出现
            \item C     \newline 
            侧面感应出负电荷,且越靠近中心电荷密度越大,相应的电场线越密集
            \item B     
            $$
            F_{n} = \frac{m v^{2}}{r} = G\frac{mM}{r^{2}} = m a_{n} \lra v^{2} = \frac{GM}{r}
            $$
            A选项同一个P点受到的万有引力一样大提供全部向心力,因此加速度一样大   \newline
            B选项调相轨道整体大于停泊轨道,在经过P点时需要万有引力大于所需的向心力,因此减速即可  \newline
            C选项,开普勒第三定律$ \frac{a^{3}}{T^{2}} = k $($k$是一个常数仅仅与中心天体有关),调相轨道的半长轴更大因此周期更大   \newline
            D选项,在运动过程中机械能守恒,不妨考虑P点引力势能一样大,调相轨道的速度更大因此动能更大,所以在此轨道上机械能更大
            \item D     \newline
            题干中的\textbf{分别}理解到位就好做了,在两种外电阻的情况下唯一不变的就是总电压,通过热量求出两种电路的电流关系再根据电压列等式可得内电阻
            \item C  \newline
            A选项,轨迹在后半段收到向下的电场力,电场方向趋势为右上,因此带\textbf{负电荷}     \newline
            B选项,后半段电场线变疏受力变小      \newline
            C选项,电荷为负电荷,显然电荷有一定初速度,且电场力持续做负功,速度一直减小     \newline 
            D选项,机械能守恒,动能在持续减小,电势能在增大
            \item C \newline 
            选项A,显然带正电    \newline 
            选项B,电场可以分解为两个垂直方向的电场$E_{BA}$与$E_{CB}$,$U_{BA} = E_{BA} L_{BA} = U_{CB} = E_{CB} L_{CB}$,带入数值即可求的$E_{BA} = \frac{1}{2}$与$E_{CB} = \frac{2}{3}$,合成后可得
            原电场强度为$ E = \frac{5}{6} V/m $     \newline 
            选项C,机械能守恒    \newline 
            选项D存疑
            \item B     \newline 
            选项A,开关S断开两条路并联,由于电容特性分得全部电压,因此可以认为$C_{1}$左侧为18V,$a$点为0V,$R_{1}$左侧为18V,$b$点为18V,$C_{2}$右侧为0V,因此$U_{ab}$为18V     \newline 
            选项B,开关断开时两个电容器均在充电,开关闭合后电容器开始放电     \newline 
            选项C,$ Q = CU $    \newline 
            选项D,闭合后放电到均不带电荷
            \item D     \newline  
            选项A,$F_{all} =  \frac{ \left(\sqrt{3} + 3 \right)  G m^{2} }{L^{2}}$    \newline 
            选项B,$ \frac{m v^{2}}{\frac{\sqrt{3}}{3} L} = F_{all} $, 算出$mv^{2}$后再算动能乘以3得到三星总动能为$ E_{k} = \frac{\sqrt{3} \left( 3 + \sqrt{3} \right) G m^{2} }{2L} $   \newline 
            选项C,$ m \omega^{2} \frac{\sqrt{3}}{3} L = F_{all} = \frac{ \left(\sqrt{3} + 3 \right)  G m^{2} }{L^{2}} \lra w^{2} \propto m $,质量变为两倍,角速度变成$\sqrt{2}$倍    \newline 
            选项D,周期计算$ T = \frac{2\pi}{\omega} $,由选项C计算出$ \omega^{2} \propto \frac{1}{L^{3}} $,所以$L$变为两倍则$\omega^{2}$变为原来的$\frac{1}{8}$,显然$ T^{2} \propto \frac{1}{\omega^{2}} $,$T^{2}$变为原来的
            8倍,开根号得到$2\sqrt{2}$

        \end{enumerate}


    \section{多选题解}
        \begin{enumerate}
            \item CD \newline 
            选项A显然错误       \newline 
            选项B,考虑整个过程中的摩擦力做功大小,$AB$路线一致因此机械能减少一样多,$C$走的路线最长,机械能损失最多.由此$B$在底端全部机械能为动能且初始机械能大于$A$,所以$B$的动能最大  \newline 
            选项C与D可由B的推理过程得出     \newline 
            \item AC \newline 
            选项A,电势能的变化存在拐点,在电势能达到最大值意味着电场力不做功,但是电子由题意并未停止过运动,因此在此处电场强度为0,那么$AB$为异种电荷,且初期收到向左的电场力后期为向右的电场力可推断$A$带负电荷,$B$带正电
            荷,随着距离变大,负电荷的作用越发明显因此它的电荷应该更大        \newline 
            选项B,电子受力向左因此电场线沿$x$轴正方向       \newline 
            选项C,$ k\frac{q_{1}q}{ \left( x_{0}+ x_{2} \right)^{2}} = k\frac{q_{2}q}{x_{2}^{2}}  \lra \frac{q_{1}}{q_{2}} = \frac{\left( x_{0} + x_{2} \right)^{2}}{x_{2}^{2}} $
            \item ABD       \newline
            $ R_{2} $ 与$ R_{3} $ 并联 再和$ R_{1} $ 串联,滑动变阻器向上滑动接入电路的电阻值变大,因此$U_{1}$变小,电路总电阻值变大$I_{1}$减小,$ R_{2} $并联电路整体分压变大此分支电流增大,因此另一分支电流减,
            所以$I_{3}$变小     \newline 
            选项D,$ \frac{\Delta U_{3}}{\Delta I_{1}}$ ,显然电路总电压从未变过,因此有$ \abs{\Delta U_{3}} = \abs{\Delta U_{1}} + \abs{\Delta U_{r}}$,因此前面的比值就是定值电阻$ R_{1} + r $
            \item CD    \newline 
            $t=0$时刻发射的粒子正好从$B$发射出去时所经过的时间由水平位移决定$ t = \frac{2d}{v_{0}} $,恰好为两个周期,竖直方向上反复进行匀加速和匀减速运动,在时间$ t = 2T $里竖直位移恰好是$d$.$ E d = \varphi _{0} $,4
            端匀加减速运动,每段时间为$\frac{1}{2} T$,$ \frac{1}{2} a \left(  \frac{1}{2} T \right)^{2} \vdot 4 = d $,得到$ \frac{1}{2} a T^{2}  = d $,带入$T = \frac{d}{v_{0}} \quad a = \frac{Eq}{m} = \frac{\varphi_{0} q}{md} $,
            注意比荷的定义是$\frac{q}{m}$,结果为$\frac{2v_{0}^{2}}{varphi_{0}}$     \newline 
            选项C,$\frac{1}{4}T$射入意味着电场力做功在$2T$时间里为0,电势能不变     \newline
            选项D,意味着竖直方向上速度不会超过$v_{0}$,电场力能做最多正功的初射时间就是$t=0$,而此时出射速度在竖直方向上为0,此选项正确 
            \item ACD   \newline
            选项A,到达$b$点时减少的机械能仅仅物体$B$收到的摩擦力做功,注意轮轴大小不一样,因此物体$B$移动的距离为绳子变长距离的一半,$ W = (10 - 8) \vdot \frac{1}{2} Mg \cos{\frac{\pi}{6}} \vdot \frac{\sqrt{3}}{3} = 10 J$     \newline 
            选项B,它们的速度比由于轮的存在,沿绳速度满足:圆环沿绳的速度为物体速度的两倍,所以$ V_{A} = V_{B} \vdot 2 \vdot \frac{10}{6}$,所以比值应该为$ 10:3 $   \newline 
            选项C,下降15米则圆环重力势能减少$90J$,此时绳子长$L = 17 m$,物体沿绳上升距离为$(17-8) \vdot \frac{1}{2} = 4.5  m$,获得重力势能$ 45J $,机械能损失$ Mg\cos{\frac{\pi}{6}} \vdot \frac{\sqrt{3}}{3} \vdot  4.5  = 45J $       \newline 
            选项D,显然正确
        \end{enumerate}

\end{document}