\documentclass{article}    

%文章相关
\usepackage[UTF8, heading = false, scheme = plain]{ctex}    %解决中文字体,不改变排版
\usepackage{geometry}                                       %调整页边距等
\usepackage{indentfirst}                                    %首行缩进
\usepackage[dvipsnames,svgnames]{xcolor}                                         %颜色包:\color{}


%图片宏包
\usepackage{graphicx}                                       %插入图片:\includegraphics{myimage.png}
\usepackage{float}
\usepackage{caption}
\usepackage{subcaption}

%数学相关
\usepackage{amsmath,amsthm}                                 %amsmath包应该在前
\usepackage{braket}                                         %狄拉克符号系统 \bra{} \ket{}

%实用的内容说明包
\usepackage{hyperref}                                       %超链接插入包:\href{url}{name}
\usepackage{multirow}                                       %插入表格用到的宏包:\begin{tabular}{|c|c|}
\usepackage{listings}                                       %插入代码等:\begin{lstlisting}[breaklines=true,backgroundcolor=\color{lightgray},title=]
\usepackage{verbatim}                                       %使用 comment 环境进行注释


%文本底纹实现
\usepackage{lipsum}                                         %该宏包是通过 \lipsum 命令生成一段本文,正式使用时不需要引用该宏包
\usepackage[strict]{changepage}                             %提供一个 adjustwidth 环境
\usepackage{framed}                                         %实现方框效果
\usepackage{newtxtext}
\usepackage{tcolorbox}                                      %文本底纹包,放在xcolor包后
                               
% environment derived from framed.sty: see leftbar environment definition
\definecolor{formalshade}{rgb}{0.95,0.95,1} % 文本框颜色
% ------------------******-------------------
% 注意行末需要把空格注释掉,不然画出来的方框会有空白竖线
\newenvironment{formal}{%
\def\FrameCommand{%
\hspace{0em}%
{\color{Green}\vrule width 0.3em}%
\colorbox{greenshade}%
}%
\MakeFramed{\advance\hsize-\width\FrameRestore}%
\noindent\hspace{-2em}% disable indenting first paragraph
\begin{adjustwidth}{}{2.5em}%
\vspace{0.11em}\vspace{0.1em}%
}
{%
\vspace{2pt}\end{adjustwidth}\endMakeFramed%
}
% ----

\definecolor{greenshade}{rgb}{0.90,0.99,0.91}               %绿色文本框,竖线颜色设为 Green
\definecolor{redshade}{rgb}{1.00,0.90,0.90}                 %红色文本框,竖线颜色设为 LightCoral
\definecolor{brownshade}{rgb}{0.99,0.97,0.93}               %莫兰迪棕色,竖线颜色设为 BurlyWood




%预设
\geometry{a4paper,left=5em,right=5em,bottom=5em,top=5em}    %设置为a4paper最好,点击pacakge geometry 查看文档
\setlength{\parindent}{2em}                                 %2em(注意不支持rem)代表每一段的首行缩进两个字符,某一行不缩进时使用 \noindent

\hypersetup{hidelinks,colorlinks=true,
linkcolor=black,urlcolor=blue}                              %对hyperref 包进行预设

\newtheorem{thm}{定理}[section]                              %定义一个新的环境 thm, 命名为定理,以 节 开始编号,数学论文中常用

\renewcommand{\proofname}{ \qquad \bf 证明}                  %更改proof为中文证明

\newenvironment{solution}{\proof[\indent \bf 解]}
{\renewcommand{\qedsymbol}{}\endproof}                      %提供解环境

\newtheorem{lemma}{引理}[section]

\newtheorem{corollary}{推论}

%一些def
\def\thmindent{\setlength{\parindent}{5em}}                  %\thmindent
\def\pfindent{\setlength{\parindent}{5.5em}}                 %\pfindent
\def\clindent{\setlength{\parindent}{4em}}                   %clindent
\def\sdr{Schr\"{o}dinger}                                    %薛定谔名字
\def\intff{\int_{-\infty}^{+\infty}}                           %积分为(-\infty,+\infty)的积分


\usepackage{latexalpha2}

%以下为前言文件引用的包以及相关用法

\begin{comment}

\usepackage[UTF8, heading = false, scheme = plain]{ctex}    %解决中文字体,不改变排版
\usepackage{geometry}					 %调整页边距等
\usepackage{indentfirst}			     %首行缩进
\usepackage{graphicx}					 %插入图片:\includegraphics{myimage.png}
\usepackage{hyperref}					 %超链接插入包:\href{url}{name}
\usepackage{multirow}					 %插入表格用到的宏包:\begin{tabular}{|c|c|}
\usepackage{xcolor} 					 %颜色包:\color{}
\usepackage{listings}					 %插入代码等:\begin{lstlisting}[breaklines=true,backgroundcolor=\color{lightgray},title=]
\usepackage{verbatim}					 %使用 comment 环境进行大段备注



\geometry{a4paper,left=5em,right=5em,bottom=5em,top=5em}    %设置为a4paper最好
\setlength{\parindent}{2em}    %2em(注意不支持rem)代表每一段的首行缩进两个字符,某一行不缩进时使用 \noindent
\hypersetup{hidelinks,colorlinks=true,linkcolor=black,urlcolor=blue}   %对hyperref 包进行预设
\newtheorem{thm}{定理}[section]     %定义一个新的环境 thm, 命名为定理,以 节 开始编号,数学论文中常用 
corollary 推论环境
lemma 引理环境
proof 证明环境
solution 解环境
center environment %居中
\textbf \emph %加粗和斜体
thm environment	%定理
$equation$ (inline) or equation environment or package amsmah %公式
itemize environment or enumerate environment %列表

\end{comment}



\title{量子力学习题集}
\author{马祥芸}

\begin{document}
    \maketitle
    \tableofcontents
    \newpage

    \section{薛定谔方程与一维定态问题}
        \subsection{一维有限势场}
        \begin{thm}\label{thm:1.1}                                                %添加\lable{书签名}进行标记,使用\pageref{书签名}添加标记位置的页码 \ref{书签名}引用定理编号
            势函数具有偶对称$V(x)=V(-x)$,$\psi(x)$和$\psi(-x)$均是波函数的解
            
            \begin{proof}
                
                $$ \frac{d^2}{[d(-x)]^2}=\frac{d^2}{dx^2} $$ 

            \end{proof}
        
        \end{thm}
    

        \begin{thm}\label{thm:1.2}
            \thmindent
            
            设$V(x)=V(-x)$,\textbf{每一个}$\psi(x)$都有确定的宇称(奇偶性)(注意每一个解的宇称可以不相同)
            
            \begin{proof}
                \pfindent
                
                由于定理\ref*{thm:1.1},构造 
                   
                    $$ f(x) = \psi(x)+\psi(-x) $$                          % $$行公式不带编号
                    $$ g(x) = \psi(x)-\psi(-x) $$
                
                $f(x)$为偶宇称,$g(x)$为奇宇称,它们均为能量$E$的解       \par  %另起段落保留缩进
                而$\psi(x)$与$\psi(-x)$都可以用$f(x)$和$g(x)$表示    

                $$ \psi(x) = \frac{1}{2} [f(x)+g(x)]  $$
                $$ \psi(-x) = \frac{1}{2} [f(x)-g(x)] $$    

           \end{proof}
           
           \begin{corollary}\label{cl:1}
                \clindent 
                
                设$v(-x) = v(x)$,而且对应于能量本征值E,方程的解无简并,则该能量本征态必有确定的宇称,例如一维 \par
                谐振子,一维对称方势阱
                
                \begin{itemize}
                    \item 若$E$非简并  \quad 本征函数具有确定宇称(两种宇称) 
                        $$ \psi(-x) = \hat{P}\psi(x) = c\psi \quad c=\pm 1 $$ %\hat{}输入算符的帽子
                    \item 若$E$简并 \quad $\psi(x)$和$\psi(-x)$分别为独立的波函数,它们的线性组合是具有宇称的解
                        $$ \psi_{\pm}(x) = \frac{1}{\sqrt{2}}[\psi(x) \pm \psi(-x)] $$
                \end{itemize}
            
           \end{corollary}

        \end{thm}
        \par
        偶宇称涉及到的函数图像如下

        \begin{figure}[H]
            \centering
            \begin{subfigure}{0.48\textwidth}
                    
                \centering
                \wolframgraphics[png]{Plot[y=Tan[x],{x,0,2.5}]}{tanx}
                \includegraphics[scale=0.4]{tanx.png}
                \caption{$y=\tan{x}$}
                
            \end{subfigure}
            \begin{subfigure}{0.48\textwidth}
                
                \centering
                \wolframgraphics[png]{Plot[y=x*Tan[x],{x,0,2.5}]}{xtanx}
                \includegraphics[scale=0.4]{xtanx.png}
                \caption{$y=x \tan{x}$}

            \end{subfigure}
              
        \end{figure}

        奇宇称涉及到的函数图像如下
        \begin{figure}[H]
            \centering
            \begin{subfigure}{0.48\textwidth}
                    
                \centering
                \wolframgraphics[png]{Plot[y=-x*Cot[x],{x,0,2.5}]}{-xcotx}
                \includegraphics[scale=0.4]{-xcotx.png}
                \caption{$y=-x \cot{x}$}
                
            \end{subfigure}
            \begin{subfigure}{0.48\textwidth}
                
                \centering
                \wolframgraphics[png]{Plot[{y=-x*Cot[x],y=x*Tan[x],y=Sqrt[4-x^2]},{x,0,2.5}]}{3plot}
                \includegraphics[scale=0.4]{3plot.png}
                \caption{三个函数曲线}

            \end{subfigure}
              
        \end{figure}

            %两个subfigure之间不能存在空行否则会出现竖排列

            
        
        由于此题的势能函数具有偶对称,因此波函数可能存在偶or奇宇称(需要分开讨论),此题中偶宇称至少存在一个交点,
        而奇宇称有解必须有条件$Q>\frac{\pi}{2}$,由题意可知存在且仅存在一个束缚态,所以保留偶宇称的唯一解即可($Q<\frac{\pi}{2}$)
        
        \subsection{一维 \texorpdfstring{$\delta$}{}势}   %使用\texorpdfstring{}{}来包括tex公式或者string需要两个参数或者仅用一个括号并用.连接
        
        先不考虑$x\neq0$的局部区域,丢掉$\delta(x)$势阱,需要用到一阶微分变化值的关系    
        
        $$ \triangle(\frac{d\psi}{dx}) = \int_{-\varepsilon}^{+\varepsilon}\frac{\hbar^2}{2\mu}V(x)\psi(x)dx $$     %\int_{}^{}   \hbar 
        
        注意不要丢了$\delta(x)$前面的参数,需要用到其积分性质
        
        $$\int_{-\varepsilon}^{+\varepsilon} \delta(x) \psi(x) dx =\psi(0) $$
        
        在归一化中,由于在$x\neq 0$其他的区域的波函数具有对称性,对其中一边积分时其值为$\frac{1}{2}$
        
        $$ \int_{0}^{+\infty}A^2 e^{-2kx} dx = \frac{1}{2} $$
        
        
    \subsection{一维分段无限深势阱}
        
      

    \section{力学量算符}


    \section{表象}


    \section{三维定态问题}

    \section{近似方法}

    \section{自旋}

    \section{全同粒子体系}

    \section{散射}
    
    


  
\end{document}