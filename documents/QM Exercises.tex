\documentclass{article}    

%文章相关
\usepackage[UTF8, heading = false, scheme = plain]{ctex}    %解决中文字体,不改变排版
\usepackage{geometry}                                       %调整页边距等
\usepackage{indentfirst}                                    %首行缩进
\usepackage[dvipsnames,svgnames]{xcolor}                                         %颜色包:\color{}


%图片宏包
\usepackage{graphicx}                                       %插入图片:\includegraphics{myimage.png}
\usepackage{float}
\usepackage{caption}
\usepackage{subcaption}

%数学相关
\usepackage{amsmath,amsthm}                                 %amsmath包应该在前
\usepackage{braket}                                         %狄拉克符号系统 \bra{} \ket{}

%实用的内容说明包
\usepackage{hyperref}                                       %超链接插入包:\href{url}{name}
\usepackage{multirow}                                       %插入表格用到的宏包:\begin{tabular}{|c|c|}
\usepackage{listings}                                       %插入代码等:\begin{lstlisting}[breaklines=true,backgroundcolor=\color{lightgray},title=]
\usepackage{verbatim}                                       %使用 comment 环境进行注释


%文本底纹实现
\usepackage{lipsum}                                         %该宏包是通过 \lipsum 命令生成一段本文,正式使用时不需要引用该宏包
\usepackage[strict]{changepage}                             %提供一个 adjustwidth 环境
\usepackage{framed}                                         %实现方框效果
\usepackage{newtxtext}
\usepackage{tcolorbox}                                      %文本底纹包,放在xcolor包后
                               
% environment derived from framed.sty: see leftbar environment definition
\definecolor{formalshade}{rgb}{0.95,0.95,1} % 文本框颜色
% ------------------******-------------------
% 注意行末需要把空格注释掉,不然画出来的方框会有空白竖线
\newenvironment{formal}{%
\def\FrameCommand{%
\hspace{0em}%
{\color{Green}\vrule width 0.3em}%
\colorbox{greenshade}%
}%
\MakeFramed{\advance\hsize-\width\FrameRestore}%
\noindent\hspace{-2em}% disable indenting first paragraph
\begin{adjustwidth}{}{2.5em}%
\vspace{0.11em}\vspace{0.1em}%
}
{%
\vspace{2pt}\end{adjustwidth}\endMakeFramed%
}
% ----

\definecolor{greenshade}{rgb}{0.90,0.99,0.91}               %绿色文本框,竖线颜色设为 Green
\definecolor{redshade}{rgb}{1.00,0.90,0.90}                 %红色文本框,竖线颜色设为 LightCoral
\definecolor{brownshade}{rgb}{0.99,0.97,0.93}               %莫兰迪棕色,竖线颜色设为 BurlyWood




%预设
\geometry{a4paper,left=5em,right=5em,bottom=5em,top=5em}    %设置为a4paper最好,点击pacakge geometry 查看文档
\setlength{\parindent}{2em}                                 %2em(注意不支持rem)代表每一段的首行缩进两个字符,某一行不缩进时使用 \noindent

\hypersetup{hidelinks,colorlinks=true,
linkcolor=black,urlcolor=blue}                              %对hyperref 包进行预设

\newtheorem{thm}{定理}[section]                              %定义一个新的环境 thm, 命名为定理,以 节 开始编号,数学论文中常用

\renewcommand{\proofname}{ \qquad \bf 证明}                  %更改proof为中文证明

\newenvironment{solution}{\proof[\indent \bf 解]}
{\renewcommand{\qedsymbol}{}\endproof}                      %提供解环境

\newtheorem{lemma}{引理}[section]

\newtheorem{corollary}{推论}

%一些def
\def\thmindent{\setlength{\parindent}{5em}}                  %\thmindent
\def\pfindent{\setlength{\parindent}{5.5em}}                 %\pfindent
\def\clindent{\setlength{\parindent}{4em}}                   %clindent
\def\sdr{Schr\"{o}dinger}                                    %薛定谔名字
\def\intff{\int_{-\infty}^{+\infty}}                           %积分为(-\infty,+\infty)的积分


\title{量子力学习题集}
\author{马祥芸}

\begin{document}
    \maketitle
    \tableofcontents
    \newpage
    %\thecoursename{量子力学习题集}  %作业包页眉左边
    %\thecourseinstitute{马祥芸}    %作业包页眉右边

    \section{薛定谔方程与一维定态问题}

        \begin{formal}
            $$ \dv[2]{y}{x} + p\dv{y}{x} + qy = 0 $$
            
            特征根方程
            $$ r^{2} + pr + q = 0 $$

            \begin{enumerate}
                \item $r_{1} \neq r_{2}$且为实根
                $$ y = Ae^{r_{1}x} + Be^{r_{2}x} $$
                \item $r_{1} = r_{2}$且为实根
                $$ y = (C_{1}+C_{2}x)e^{r_{1}x} $$
                \item $r_{1}=\alpha + i \beta,r_{2} = \alpha - i \beta $为共轭复根(通常$\alpha$都是0)
                $$ y = e^{\alpha x} (C_{1}\cos{\beta x} + C_{2}\sin{\beta x}) \quad or \quad y = C_{1}e^{i\beta x} + C_{2}e^{-i\beta x}$$

                一阶微分变化值的关系
                $$ \triangle(\frac{d\psi}{dx}) = \int_{-\varepsilon}^{+\varepsilon}\frac{\hbar^2}{2\mu}V(x)\psi(x)dx $$    

                $\delta$函数性质

                $$\int_{-\varepsilon}^{+\varepsilon} \delta(x) \psi(x) dx =\psi(0) \quad \delta(Rx) = \frac{1}{R} \delta(x) $$

                常见三角函数的公式
                $$\sin(x+y)=\sin{x} \cos{y}+\sin{y}\cos{x} \quad \cos(x+y)=\cos{x} \cos{y} - \sin{x}\sin{y}$$ 
                $$\sin(2x) = 2\sin{x} \cos{x} \quad \cos(2x) = \cos^{2}{x} - \sin^{2}{x} $$
                $$\sin^{2}{x}=\frac{1-\cos(2x)}{2} \quad \cos^{2}{x}=\frac{1+\cos(2x)}{2} $$
                $$ 1 + \tan^{2}{x} = \frac{1}{\cos^{2}{x}} \quad \dv{\tan{x}}{x} = \frac{1}{\cos^{2}{x}} $$

                两个常见积分

                $$ \intff e^{-a (x+b)^{2}} dx = \sqrt{\frac{\pi}{a}} $$
                $$ \intff x^{2} e^{-a x^{2}} dx = \frac{1}{2a} \sqrt{\frac{\pi}{a}} $$

                动量表象问题
                $$ \psiii{p}{x} = \frac{1}{(2 \pi \hbar)^{\frac{n}{2}}} e^{\frac{ipx}{\hbar}} $$ 
                $$ \psiii{x}{p} = \frac{1}{(2 \pi \hbar)^{\frac{n}{2}}} e^{\frac{-ipx}{\hbar}} $$ 
                $$ \delta(x) = \frac{1}{(2\pi)^{n}} \intff e^{ipx} dp $$
                $$ \delta(p) = \frac{1}{(2\pi)^{n}} \intff e^{-ipx} dx $$
                $$ \psi(p) = \braket{p}{\psi} = \int dx' \braket{p}{x'} \braket{x'}{\psi} = \frac{1}{(2 \pi \hbar)^{\frac{n}{2}}} \intff \psi(x) e^{\frac{-ipx}{i\hbar}} dx $$
                $$ \psi(x) = \braket{x}{\psi} = \int dp' \braket{x}{p'} \braket{p'}{\psi} = \frac{1}{(2 \pi \hbar)^{\frac{n}{2}}} \intff \psi(p) e^{\frac{ipx}{i\hbar}}  dp $$

                海森堡绘景
                $$ \dv{A_{H}(t)}{t} = \pdv{A_{H}(t)}{t} + \frac{1}{i\hbar} [A_{H}(t),H] $$
                $$ \dv{<A_{H}(t)>}{t} = \pdv{<A_{H}(t)>}{t} + \frac{1}{i\hbar} \overline{[A_{H}(t),H]} $$

                概率流密度
                $$ j_{x} = \frac{1}{2} (\psi^{*} \vu*{v} \psi - \psi \vu*{v} \psi^{*} ) \quad \vu*{v} = \frac{\vu*{p}}{\mu} = - \frac{i\hbar}{\mu} \pdv{x} $$
                $$ T = \abs{\frac{j_{R}}{j_{I}}} \quad R = \abs{\frac{j_{T}}{j_{I}}} $$

                无限深方势阱([0,a]),归一化系数$A$,势阱长度$L$.
                $$ \psiii{n}{x} = \sqrt{\frac{2}{a}} \sin{\frac{n\pi x}{a}} \quad \frac{1}{2} L A^{2}  = 1 $$
                $$ E_{n} = \dfrac{n^{2}\pi^{2}\hbar^{2}}{2\mu a^{2}} $$

                谐振子
                $$ \psiii{n}{\xi} = N_{n} H_{n}(\xi) e^{-\frac{\xi^{2}}{2}} $$
                $$ N_{n} = \sqrt{\frac{\alpha}{\sqrt{\pi}2^{n} n!}}  \quad \alpha = \sqrt{\frac{\mu \omega}{\hbar}} \quad \xi = \alpha x $$
                $$ \psiii{0}{\xi} = \sqrt{\frac{\alpha}{\sqrt{\pi}}} e^{-\frac{\xi^{2}}{2}} \quad (H_{0}(\xi) = 1)$$
                $$ \psiii{1}{\xi} = \sqrt{\frac{\alpha}{\sqrt{\pi}}} \sqrt{2} \xi e^{-\frac{\xi^{2}}{2}} \quad (H_{1}(\xi) = 2\xi) $$
                
                $\qquad$谐振子波函数具有宇称$(-1)^{n}$,通常用于奇函数的积分性质,根据维里定理
                $$  < T > \quad = \quad  < V >  $$
                
                $\qquad$能量与升降算符
                $$ E_{n} = (\vu*{N} + \frac{1}{2}) \hbar \omega = (n + \frac{1}{2}) \hbar \omega   \quad  \vu*{N} = \ua \da   $$
                $$ \vu*{a}_{\pm} = \frac{1}{\sqrt{2\mu \hbar \omega}} (\mp i\vu*{p} + x) \quad \comm{\da}{\ua} = 1 $$
                $$ \ua \ket{n} = \sqrt{n+1}\ket{n+1} \quad \da\ket{n} = \sqrt{n}\ket{n-1} \quad \vu*{N}\ket{n} = n \ket{n}$$
                
                $\qquad$交叉相谐振子处理方式
                $$ \xi = \frac{1}{\sqrt{2}} (x+y) \quad \eta = \frac{1}{\sqrt{2}} (x -y)  $$
                $$ \xi^{2} + \eta^{2} = x^{2} + y^{2} \quad xy = \frac{1}{2} (\xi^{2} - \eta^{2}) \quad \pdv[2]{x} + \pdv[2]{y} = \pdv[2]{\xi} + \pdv[2]{\eta}$$
                
                $\qquad$其他性质
                $$  
                x \psi_{n} = \frac{1}{\alpha}(\sqrt{\frac{n}{2}}\psi_{n-1} + \sqrt{\frac{n+1}{2}} \psi_{n+1})  \quad 
                \dv{\psi_{n}}{x} = \alpha (\sqrt{\frac{n}{2}}\psi_{n-1} - \sqrt{\frac{n+1}{2}} \psi_{n+1}) 
                $$



                
                

                
            \end{enumerate}
        \end{formal}

        \subsection{一维有限势场}
        \begin{thm}\label{thm:1.1}                                                %添加\lable{书签名}进行标记,使用\pageref{书签名}添加标记位置的页码 \ref{书签名}引用定理编号
            势函数具有偶对称$V(x)=V(-x)$,$\psi(x)$和$\psi(-x)$均是波函数的解
            
            \begin{proof}
                
                $$ \frac{d^2}{[d(-x)]^2}=\frac{d^2}{dx^2} $$ 

            \end{proof}
        
        \end{thm}
    

        \begin{thm}\label{thm:1.2}
            \thmindent
            
            设$V(x)=V(-x)$,\textbf{每一个}$\psi(x)$都有确定的宇称(奇偶性)(注意每一个解的宇称可以不相同)
            
            \begin{proof}
                \pfindent
                
                由于定理\ref{thm:1.1},构造 
                   
                    $$ f(x) = \psi(x)+\psi(-x) $$                          % $$行公式不带编号
                    $$ g(x) = \psi(x)-\psi(-x) $$
                
                $f(x)$为偶宇称,$g(x)$为奇宇称,它们均为能量$E$的解       \par  %另起段落保留缩进
                而$\psi(x)$与$\psi(-x)$都可以用$f(x)$和$g(x)$表示    

                $$ \psi(x) = \frac{1}{2} [f(x)+g(x)]  $$
                $$ \psi(-x) = \frac{1}{2} [f(x)-g(x)] $$    

           \end{proof}
           
           \begin{corollary}\label{cl:1}
                \clindent 
                
                设$V(-x) = V(x)$,而且对应于能量本征值E,方程的解无简并,则该能量本征态必有确定的宇称,例如一维 \par
                谐振子,一维对称方势阱
                
                \begin{itemize}
                    \item 若$E$非简并  \quad 本征函数具有确定宇称(两种宇称) 
                        $$ \psi(-x) = \hat{P}\psi(x) = c\psi \quad c=\pm 1 $$ %\hat{}输入算符的帽子
                    \item 若$E$简并 \quad $\psi(x)$和$\psi(-x)$分别为独立的波函数,它们的线性组合是具有宇称的解
                        $$ \psiii{\pm}{x} = \frac{1}{\sqrt{2}}[\psi(x) \pm \psi(-x)] $$
                \end{itemize}
            
           \end{corollary}

        \end{thm}
        \par
        偶宇称涉及到的函数图像如下

        \begin{figure}[H]
            \centering
            \begin{subfigure}{0.48\textwidth}
                    
                \centering
                \wolframgraphics[png]{Plot[y=Tan[x],{x,0,2.5}]}{tanx}
                \includegraphics[scale=0.4]{tanx.png}
                \caption{$y=\tan{x}$}
                
            \end{subfigure}
            \begin{subfigure}{0.48\textwidth}
                
                \centering
                \wolframgraphics[png]{Plot[y=x*Tan[x],{x,0,2.5}]}{xtanx}
                \includegraphics[scale=0.4]{xtanx.png}
                \caption{$y=x \tan{x}$}

            \end{subfigure}
              
        \end{figure}

        奇宇称涉及到的函数图像如下
        \begin{figure}[H]
            \centering
            \begin{subfigure}{0.48\textwidth}
                    
                \centering
                \wolframgraphics[png]{Plot[y=-x*Cot[x],{x,0,2.5}]}{-xcotx}
                \includegraphics[scale=0.4]{-xcotx.png}
                \caption{$y=-x \cot{x}$}
                
            \end{subfigure}
            \begin{subfigure}{0.48\textwidth}
                
                \centering
                \wolframgraphics[png]{Plot[{y=-x*Cot[x],y=x*Tan[x],y=Sqrt[4-x^2]},{x,0,2.5}]}{3plot}
                \includegraphics[scale=0.4]{3plot.png}
                \caption{三个函数曲线}

            \end{subfigure}
              
        \end{figure}

            %两个subfigure之间不能存在空行否则会出现竖排列

            
        
        由于此题的势能函数具有偶对称,因此波函数可能存在偶or奇宇称(需要分开讨论),此题中偶宇称至少存在一个交点,
        而奇宇称有解必须有条件$Q>\frac{\pi}{2}$,由题意可知存在且仅存在一个束缚态,所以保留偶宇称的唯一解即可($Q<\frac{\pi}{2}$)
        
        \subsection{一维 \texorpdfstring{$\delta$}{}势}   %使用\texorpdfstring{}{}来包括tex公式或者string需要两个参数或者仅用一个括号并用.连接
        
        先不考虑$x\neq0$的局部区域,丢掉$\delta(x)$势阱,需要用到  
        \begin{formal}
            
            \indent 一阶微分变化值的关系  

            $$ \triangle(\frac{d\psi}{dx}) = \int_{-\varepsilon}^{+\varepsilon}\frac{\hbar^2}{2\mu}V(x)\psi(x)dx $$     %\int_{}^{}   \hbar 

            积分性质

            $$\int_{-\varepsilon}^{+\varepsilon} \delta(x) \psi(x) dx =\psi(0) $$

        \end{formal}
        
        
        注意不要丢了$\delta(x)$前面的参数
        
        
        在归一化中,由于在$x\neq 0$其他的区域的波函数具有对称性,对其中一边积分时其值为$\frac{1}{2}$
        
        $$ \int_{0}^{+\infty}A^2 e^{-2kx} dx = \frac{1}{2} $$
        
        
    \subsection{一维分段无限深势阱}
        此题的特点是$x=0$处的$V(x)|_{x=0}=\infty$,与$\delta(x)$势不一样的是,
        虽然在此处的势能大小都是为$\infty$,但是前者的$\psi(0)=0$(也因此$\triangle \frac{d\psi}{dx}$=0,连续)而后者并不为$\psi(0)\neq0$,
        所以$\delta(x)$势通常在此点并不连续。

        当然由于$V(x)$具有偶对称性,波函数同样具有确定的宇称,现假设两个排除0点的波函数解分别为
        $$\psiii{1}{x}=B\sin(kx) \quad (0<x<a) \quad \psiii{2}{x}=D\sin(kx) \quad (-a<x<0)$$
        给两种方法通过宇称判断系数关系
        \begin{itemize}
            \item 全局判断法   \\
                若$\psi(x)$在$|x|<a$上为奇宇称,那么恰好为正弦函数$\sin(kx)$(奇函数)的形式 $\Rightarrow$ $B = D$ 
            \item 定义法    \\
                由奇宇称的定义$\psi(x)=-\psi(-x) \Rightarrow B\sin(kx)=-D\sin(-kx)=D\sin(kx) \Rightarrow B=D$ 
        \end{itemize}
        
        最后需要注意$n$的取值范围,应该是从$n=1,2,3\cdots$不能从0开始因为$ka=0 \Rightarrow k=0$(能量为0)
        
        可能在归一化中需要用到的三角函数数学公式
        $$\sin(x+y)=\sin{x} \cos{y}+\sin{y}\cos{x} \quad \cos(x+y)=\cos{x} \cos{y} - \sin{x}\sin{y}$$ 
        $$\sin(2x) = 2\sin{x} \cos{x} \quad \cos(2x) = \cos^{2}{x} - \sin^{2}{x} $$
        $$\sin^{2}{x}=\frac{1-\cos(2x)}{2} \quad \cos^{2}{x}=\frac{1+\cos(2x)}{2} $$

    \subsection{半壁无限深势阱}
        再次遇到$y=-x \cot{x}$,记忆关键点的方式可以通过极限来记忆
        $$\lim_{x \to 0}- x \cot{x} = \lim_{x \to 0} -\frac{x}{\sin{x}} + \lim_{x \to 0}\cos{x} = -1 \times 1 = -1  $$
        $$\lim_{x \to \frac{\pi}{2}}- x \cot{x} = \lim_{x \to \frac{\pi}{2}} -\frac{x}{\sin{x}} + \lim_{x \to \frac{\pi}{2}}\cos{x} = -\frac{\pi}{2} \times 0 = 0  $$
        
        最后在此题中可变参量为$a$与$V_{0}$,最好化简为不等式一边仅有可变参量,正如
        $$V_{0}a^{2}\geq \frac{\pi^{2}\hbar^{2}}{8 \mu} $$

    \subsection{复合势:\texorpdfstring{$\delta(x)$}{}和阶梯势}
        注意任何含有$\delta(x)$的势场其束缚态能量必然是负数,所以$E<0$,明确这一点再求解,同样$x=0$处波函数不连续,在求一阶微分关系时不要忘记$\delta(x)$前面的的所有系数
        此题束缚态条件比较特殊,是可解析的等式,不需要两个方程联立作图求解,最后保证一方为根式,另一方包含所有可变参量并要求$>0$即可.同时在最后的
        归一化过程中需要全空间积分为1(不是对称函数).

    \subsection{复合势:\texorpdfstring{$\delta(x)$}{}和阶梯势}
        此题直接带入波函数的连续性条件得到的方程组是难以求解的,因此需要特殊技巧(两部分解分别满足连续性和一阶微分连续性)
        \begin{itemize}
            \item 获得奇宇称的解,满足一阶微分连续性,无视$\delta(x)$势,采用无限深方势阱的解,只取在$x=\frac{a}{2}$的有效解(此处为0的解)
            \item 获得偶宇称解重点在于$\psiii{2}{a}=0$,所以不妨让$\psiii{2}{x}=A \sin{(x-a)}$,同时在$x=\frac{a}{2}$处连续得到$\psi(x)=-A \sin{(x-a)}$,它是很容易验证在关于$x=\frac{a}{2}$对称的.(设对称轴为$x=b$)
                $$\psiii{1}{b-x}=\psiii{2}{b+x} \quad \Rightarrow b=\frac{a}{2}$$
        \end{itemize}

        注意在求第一激发态的时候还没有考虑$a \to 0$,所以对于偶宇称的解的最低能量是在某一个区间,需要把两种宇称解的最低能量进行对比.

    \subsection{复合势:\texorpdfstring{$\delta(x)$}{}和谐振子势}
        加入$\delta(x)$后需要重新考虑$x=0$的一阶连续情况,也就是$\psi(0)$的值,若$\psi(0)=0$则原来的解仍成立反之不成立,所以带入$x=0$后发现是$H(0)=0$
        即可,事实上仅有$n=1,3,5\cdots$成立

    \subsection{反比例势:合流超几何函数}\label{subsec:1.8}
        关键点
        \begin{itemize}
            \item 整理微分方程形如 \quad $\frac{d^{2}\psi(x)}{d x^{2}}-k^{2}\psi(x)+\frac{\beta}{x}\psi(x)$
            \item 带入$\psi(x)=x e^{-kx}F(x)$进一步整理微分方程
            \item 变量代换$\xi \to 2kx$ 进一步整理微分方程
            \item 形如\quad $\xi \frac{d^{2}F(\xi)}{d \xi^{2}}+(\gamma-\xi)\frac{dF(\xi)}{d \xi}-\alpha F(\xi)=0$  
                  $$ E=-|E| \quad \beta=\frac{2\mu a}{\hbar^{2}} \quad \gamma=2 \quad \alpha=1-\frac{\beta}{2k}=1-\frac{\mu a}{k \hbar^{2}}$$ 
                  $$ \psi(\xi)=A\xi e^{\frac{-\xi}{2}} F(\alpha,\gamma,\xi)   $$
        \end{itemize}
        一般不考,记得反比例势能的解和合流超几何方程有关就行了,其解为合流超几何函数,此题和1.9,1.10的差不多
    
    \subsection{氢原子势能}
        见\ref{subsec:1.8}题

    \subsection{反比例势能}
        见\ref{subsec:1.8}题

    \subsection{已知波函数与\texorpdfstring{V(x)}{}的极限}\label{subsec:1.11}
        此题具有启发性,当已知波函数时,那么波函数的二阶导数同样已知,因此\sdr 方程的未知数仅有$V(x)$与$E$,可以得到$V(E,x)$方程,在利用
        额外条件进行求解,此题为$x \to +\infty \quad V \to 0$,可以解得$E$,再求解$V(x)$ 

        求导的时候需要小心,此题的二阶导一共有4项
    
    \subsection{已知波函数与\texorpdfstring{V(x)}{}的均值}
        同题目\ref{subsec:1.11}类似,不过给出另一个已知条件是$\bra{\psi}V\ket{\psi}=0$,记住利用这类已知条件时不要贸然带入波函数进行求解,应该
        凑题目条件,同时获得一个经验就是能量$E$是与坐标变量无关的,通常是优先求的,其次在得到$\int \psi^{*}E\psi dx$后不要变成$\bar{E}$,能量的平均值和
        定态能量并不是同一个东西.

    \subsection{已知能量与势能的关系}
        求解过程中注意三角函数的周期性  
        $$  \arctan{(-1)}=-\frac{\pi}{4}+n\pi \quad (n=1,2,3 \cdots ) $$    
    
    \subsection{已知两能量的本征态}
        此题的关键点
        \begin{itemize}
            \item 两个有能量的本征态具有正交性
                  $$ \int_{-\infty}^{\infty} \psiiii{1}{*}{x} \psiii{2}{x}  = 0 $$
        \end{itemize}

        但是直接利用以上正交关系来直接求得$b,c$是复杂又难以实现的,我们需要额外的关系来先求得一个参数化简第二个波函数. 

        由于$\psiii{1}{x}$的信息是完全可知的,因此我们需要利用它来获得关于$V(x)$的信息,本题可得到$V(x)$具有偶对称性,因此我们可以化简$\psiii{2}{x}$,
        只能存在一个偶宇称即$b=0$.

        这个积分可拆分成如下两个积分
            $$ \intff c e^{-\beta x^{2}} dx  \quad + \quad   \intff x^{2} e^{-\beta x^{2}} dx=0 \quad (1)$$   
        
        
        \begin{formal}
        这两个积分相当典型,在后面使用高斯试探函数经常会遇到此类积分,现总结
        

        \begin{enumerate}
            \item 
                $$ \intff e^{-a (x+b)^{2}} dx = \sqrt{\frac{\pi}{a}} $$
            \begin{proof}
                \pfindent 

                $$I=\intff e^{-a x^{2}} dx$$
                $$I^{2}= \intff e^{-a (x^{2}+y^{2})} dx dy $$

                令$x=r\cos{\theta} \quad y=r\sin{\theta}$
                $$ I^{2} = \int_{0}^{+\infty} \int_{0}^{2\pi} e^{-a r^{2}} rdrd\theta $$
               
                \begin{align*}
                    I^{2} &= \int_{0}^{+\infty} \pi  e^{-a r^{2}}d(r^{2})   \\
                          &= \frac{\pi}{a}  \lra I = \sqrt{\frac{\pi}{a}} 
                \end{align*}
            
            \end{proof}

            \item 
                $$ \intff x^{2} e^{-a x^{2}} dx = \frac{1}{2a} \sqrt{\frac{\pi}{a}} $$

                特别的当$n=1,3,5,7\cdots$
                $$ \intff x^{n} e^{-a x^{2}} dx = 0 $$

                \begin{proof}
                    \pfindent
                    $$ \frac{d(e^{-a x^{2}})}{dt} = -2a x e^{-a x^{2}} $$
                    
                    \begin{align*}
                        I &= -\frac{1}{2a} \intff x d(e^{-a x^{2}})                                         \\
                          &= -\frac{1}{2a} (\eval{x e^{-a x^{2}}}_{-\infty}^{+\infty} - \intff e^{-a x^{2}}dx )
                    \end{align*}

                    洛必达法则
                    $$ \lim_{x \to \pm \infty} x e^{-x^{2}} = \eval{\frac{1}{2x e^{x^{2}}}}_{x \to \pm \infty} = 0^{+} \quad and \quad 0^{-} $$ 
                    $$ I = \frac{1}{2a} \sqrt{\frac{\pi}{a}}$$

                \end{proof}

            \item 表格分部积分法(处理复杂分部积分函数):被积函数的结构为---(多项式)(函数)\quad(本质是分部积分)
                $$ \int(x^{2}+x) e^{3x} dx \quad \int x(x-a)\sin{2x} \quad \int e^{3x}\sin{2x}dx $$
                记为$f(x) \quad g(x)$

              
                \begin{center}

                    \begin{tabular}{|c|c|c|c|c|}
                        
                        \hline
                        $f(x)$ & $f'(x)$        & $f''(x)$          & $\cdots$ & 0 \\
                        \hline
                        $g(x)$ & $\int g(x) dx$ & $\iint g(x) dxdx$ & $\cdots$ & $\iiint \dots$ \\
                        \hline
                        
                    \end{tabular}
                    
                \end{center}
                
                第二行的项数与第一行保持一致,共计[n,n]

                $+(1,2)-(2,3)+(3,4)-(4,5)\cdots + c$ \quad 
                
                注意:(i,i+1)表示第一行第$i$个元素$\times$第二行的第$(i+1)$个元素,每一个乘积前的正负号为[+,-,+,...]交替,同时不要漏掉积分常数$c$,如果
                第一行的函数无法求导到0,求导直到出现原函数的常数倍也可以.($\int e^{3x}\sin{2x}dx $的积分第一行第三项与第二行第三项积的积分为原函数的$-\frac{9}{4}$倍),
                减去一个$A I$(A为常数,前一个例子中为$\frac{-9}{4}$),移项即可
            \end{enumerate}   
        \end{formal}

        回到原积分$I_{1}+I_{2}=0$,第一个积分值很容易知道为$c \sqrt{\frac{\pi}{\beta}}$,第二个积分值为$\frac{1}{2\beta} \sqrt{\frac{\pi}{\beta}}$,求得$ c = -\frac{1}{2\beta}$

        方程(9)带入波函数求解复杂,需要细心,其中有一部需要分解因式(具有启发性,二阶导为原函数的一个多项式倍)
        \begin{align*}
            \frac{\psiiii{2}{''}{x}}{\psiii{2}{x}} & = \frac{\beta (2\beta^{2} x^{4} -11\beta x^{2} +5)}{2\beta x^{2}-1} \\
                                        & = \frac{\beta (2\beta x^{2} - 1) (\beta x^{2}-5)}{2\beta x^{2}-1}   \\
                                        & = \beta(\beta x^{2}-5) 
        \end{align*}
        
        \subsection{圆圈运动}

        此题的$x$是以圆环的外周长为度量的,需要变换波函数的变量便于求解$x=R\varphi$,因此$\frac{d}{dx^{2}}=\frac{1}{R^{2}d\varphi^{2}}$

        此时$V(x)=a\delta(x-L/2) \lra V(\varphi) = a \delta[R(\varphi-\pi)]$
        值得注意的一个$\delta(x)$的缩放性质
        \begin{formal}
            $$ \intff \delta(Rx)dx = \intff \frac{1}{R} \delta(Rx) d(Rx) = 1 \lra \delta(Rx) = \frac{1}{R}\delta(x)$$
        \end{formal}

        所以我们得到新的势函数$V(\varphi) = \frac{a}{R} \delta(\varphi-\pi) $,在求解过程中不使用三角函数解,使用复幂指数的解更合适(涉及角度),\quad $\psi(x) = Ae^{-ik\varphi} + Be^{ik\varphi}$

        连续性条件发生变化,发散点为$\varphi = \pi$,实际上第三个条件和第一个条件给出的结论是一样的,而第二个条件往往是被忽略的
        $$ \psi_{1}(0) = \psi_{2}(2\pi) \quad \psi_{1}'(0) = \psi_{2}'(2\pi)\quad \psi_{1}(\pi) = \psi_{2}(\pi) $$
        
        由前两个条件可以得到如下两个方程组
        \begin{align}
            A+B&=C+D\\
            A-B&=C-D
        \end{align}

        容易解出$A=C$带入方程$(1)or(2)$会得到$B=D$,$A$与$B$的关系需要一阶波函数在$\varphi=\pi$的连续性关系解出,之后我们需要再将复幂指数的解在返回三角函数形式并
        归一化得到
        $$ \psi(x) = \sqrt{\frac{1}{\pi}} \sin{m \varphi} $$
        
        存在一个隐藏的周期性边界条件限制$m$的取值
        $$ \psi(\varphi) = \psi(\varphi+2\pi) \lra 2m\varphi = 2n\pi \quad (n=1,2,3,4\cdots) \lra m = 1,2,3,4\cdots $$
        
        由此我们可以反解出
        $$ E_{m} = \frac{\hbar^{2} m^{2}}{2\mu R^{2}} \quad (m=1,2,3,4\cdots)$$

        \subsection{改变哈密顿量求本征值(表象变换)}
            此题的关键在于表象的变换,由坐标表象转化到动量表象(详见曾书$P_{151}$和$P_{281-6}$)
            
            \begin{formal}
                $$ \vu*{x} = i\hbar \pdv{\vu*{p}}$$

                \begin{proof}
                    \pfindent
    
                    $$ \psiii{p}{x} = \frac{1}{(2 \pi \hbar)^{\frac{n}{2}}} e^{\frac{ipx}{\hbar}} $$ 
                    $$ \psiii{x}{p} = \frac{1}{(2 \pi \hbar)^{\frac{n}{2}}} e^{\frac{-ipx}{\hbar}} $$ 
                    $$ \delta(x) = \frac{1}{(2\pi)^{n}} \intff e^{ipx} dp $$
                    $$ \delta(p) = \frac{1}{(2\pi)^{n}} \intff e^{-ipx} dx $$
    
                    n为维数,这里取1进行证明,证明前须知
                    
                    内积$\braket{x}{\psi}$就是波动力学的波函数
                    $$\psi (x) \xlongequal{def} \braket{x}{\psi}$$
                    
                    进一步可知动量在坐标表象下即为动量波函数
                    $$ \braket{x}{p} = \frac{1}{(2 \pi \hbar)^{\frac{n}{2}}} e^{\frac{ipx}{\hbar}} $$
                    $$ \braket{p}{x} = \frac{1}{(2 \pi \hbar)^{\frac{n}{2}}} e^{\frac{-ipx}{\hbar}} $$
    
                    算符$\vu*{x}$在坐标表象下的形式为$x$,同理算符$\vu*{p}$在动量表象下为$p$
                    $${\vu*{x}} \ket{x}= x \ket{x} \quad {\vu*{p}} \ket{p} = p \ket{p} $$
    
                    关于$\delta$函数
                    $$ \braket{x'}{x} = \delta(x'-x) $$
    
                    坐标算符在自己坐标表象下的矩阵元
                    $$ x_{x'x''} = \mel{x'}{\vu*{x}}{x''} = x' \braket{x'}{x''} = x' \delta(x'-x'') $$
    
                    \begin{align*}
                        x_{p'p''} &= \mel{p'}{x}{p''} \\
                                &= \braket{p'}{x'} \mel{x'}{x}{x''} \braket{x''}{p''}                                                                  \\
                                &= \frac{1}{(2 \pi \hbar)} \iint e^{\frac{-ip' x'}{\hbar}}  e^{\frac{ip''x''}{\hbar}} x' \delta(x'-x'') dx'dx''          \\
                                &= \frac{1}{(2 \pi \hbar)} \int x' e^{\frac{-ix(p'-p'')}{\hbar}}   dx'                                                   \\
                                &= \frac{1}{(2 \pi)} \int x' e^{-i(p'-p'')\frac{x}{\hbar}}   d(\frac{x'}{\hbar})
                    \end{align*}
    
                    积分内恰好出现了一个$x'$也就是坐标算符
                    $$ \dv{e^{\frac{-ix(p'-p'')}{\hbar}}}{p'} = -\frac{i}{\hbar} e^{\frac{-ix(p'-p'')}{\hbar}}$$
                    $$ e^{\frac{-ix(p'-p'')}{\hbar}} = i\hbar \dv{}{p'} e^{\frac{-ix(p'-p'')}{\hbar}}$$
                    
                    因此
                    \begin{align*}
                         \frac{1}{(2 \pi)} \int x' e^{\frac{-ix(p'-p'')}{\hbar}}   dx' &=  \frac{1}{(2 \pi)}  i\hbar \dv{}{p'} 2\pi \delta(p'-p'')    \\
                                                                                      &= i \hbar \dv{}{p'} \delta(p'-p'')                          \\                                   
                    \end{align*}
    
                    有了矩阵元后,考虑算符的一般作用
                    $$ \ket{\varphi} = \vu*{x} \ket{\psi} \lra \braket{p}{\varphi}=\mel{p}{\vu*{x}}{\psi} \lra \varphi_{p} = \int dp' x_{pp'} \psi_{p'}$$
                    $$ \varphi_{p'} =  \int dp'' [x_{p'p''}] \psi_{p''} = \int dp'' [i \hbar \dv{}{p'} \delta(p'-p'')] \psi_{p''} = i\hbar \dv{}{p'} \psi_{p'} $$
    
                    
                
                \end{proof}
            \end{formal}
            


        \subsection{期望值问题:海森堡绘景}
            此题涉及到两种绘景的选择:薛定谔绘景和海森堡景
            
            \begin{formal}

                \begin{itemize}
                    \item 
                    \textbf{薛定谔绘景} 

                    此绘景下,负责时间演化的算符是一种幺正算符($ UU^{*} = U^{*}U = I_{n} \quad U^{-1} = U^{*} $),态向量$\ket{\psi(0)}_{s}$,经过时t,演化到$\ket{\psi(t)}_{s}$,演化方程表示为
    
                    $$ \ket{\psi(t)}_{s} = U(t,0) \ket{\psi(0)}_{s} $$
    
                    $U(t,0)$是时间从0流易到t的时间演化算符(或者写为时间$t_{0}$),是幺正算符,假设系统哈密顿量$H$不含时间,则时间演化算符为
    
                    $$ U(t,0) = e^{\frac{-iHt}{\hbar}} $$
    
                    而且时间演化算符与哈密顿量对易,注意指数函数$e^{\frac{-iHt}{\hbar}}$必须通过泰勒级数进行计算

                    \item   
                    \textbf{海森堡绘景} 

                    态向量$\ket{\psi(t)}_{H}$,算符$A_{H}(t)$的定义分别为
                    $$ \ket{\psi(t)}_{H} \xlongequal{def} \ket{\psi(0)}_{H} = \ket{\psi(0)}_{s}$$
                    $$ A_{H}(t) \xlongequal{def} U^{\dagger}(t,0)A_{s}U(t,0) $$

                    时间演化算符对时间的偏导数为
                    $$ \pdv{U(t,0)}{t} = \frac{1}{i\hbar} H U(t,0) $$
                    $$ \pdv{U^{\dagger}(t,0)}{t} = - \frac{1}{i\hbar} U^{\dagger}(t,0) H$$

                    所以算符$A_{H}(t)$对时间的导数为
                    $$ \dv{A_{H}(t)}{t} = \frac{1}{i \hbar} [U^{\dagger} A_{s} U,U_{\dagger} H U]$$

                    不含时间的哈密顿量在两种绘景下完全一样
                    $$ H_{H} = U^{\dagger} H_{s} U = H_{s} =H $$

                    将算符的定义纳入考虑,得到海森堡运动方程
                    $$ \dv{A_{H}(t)}{t} = \frac{1}{i\hbar} [A_{H}(t),H] $$
                    $$ \dv{<A_{H}(t)>}{t} = \frac{1}{i\hbar} \overline{[A_{H}(t),H]} $$
                \end{itemize}

                宁外在解题过程中需要用到一个特殊的对易关系
                $$ [\vu*{x} , F(\vu*{p})] = i\hbar F'(\vu*{p}) \llra [\vu*{x} , \vu*{p}^{n}] = i\hbar n \vu*{p}^{n-1} $$
                $$ [\vu*{p} , F(\vu*{x})] = -i\hbar F'(\vu*{x}) \llra [\vu*{p} , \vu*{x}^{n}] = -i\hbar n \vu*{x}^{n-1} $$
                


                
            \end{formal}
            

        \subsection{转子势能突变}
            自由转子和自由粒子的解的形式相似
            $$ \psi = A e^{-imx} + Be^{imx} $$
            
            通常两个传播方向会将其合并
            $$ \psi_{m} = A e^{imx} $$ 

            但是自由转子具有周期性边界条件$ \psi(x) = \psi(x+2\pi) $因此使得$m$的取值只有整数$ m = 0,\pm{1},\pm{2},\pm{3}\cdots $,也正因为是分立指标,所以
            和自由粒子有所不同,可以简单的写成求和.
            
            任何波函数都可以由它进行线性组合
            组合而成
            $$ \psi(\varphi,t) = \sum_{m} c_{m} \psiii{m}{\varphi} U(t,0 ) $$

            所以题目要求我们求出处于新的能量基态概率$\abs{c_{0}}^{2}$,因此我们先要求出$c_{m}$,事实上它是由初始条件决定的(初始波函数)

            同样的在我们已知了初始波函数与初始能量,初始波函数仍然可以用$\psiii{m}{\varphi}$展开(新解具有完备性可以组合任何波函数)($t=0,U(0,0)=1$)
            $$ \psi(\varphi,0) = \sum_{m} c_{m} \psi_{m}(\varphi) $$
            $$ \psiiii{n}{*}{\varphi} \psi(\varphi,0) = \sum_{m} c_{m} \psiiii{n}{*}{\varphi} \psiii{m}{\varphi} $$

            对其进行积分,只留下了$c_{m}$项进行积分
            $$ \int \psiiii{m}{*}{\varphi} \psi(\varphi,0) d\varphi = \int c_{m} \psiiii{m}{*}{\varphi} \psiii{m}{\varphi} d\varphi $$
            $$ c_{m} =  \int_{0}^{\varphi_0} \psiiii{m}{*}{\varphi} \psi(\varphi,0) d\varphi$$
            
            令$m=0$
            $$ c_{0} = \int_{0}^{\varphi_0} \sin{\frac{\pi \varphi}{\varphi_{0}}} d\varphi  = \frac{2\varphi_{0}}{\pi \sqrt{\phi \varphi_{0}}} $$
            $$ \abs{c_{0}}^{2} = \frac{4 \varphi_{0}}{\pi^{3}}  $$

            时间演化算符并不影响粒子处于某个态的概率,因此当移除壁垒后概率仍旧以移除前的波函数作为初始状态(初始条件),这样将初始波函数展开(移除后的波函数可解),一些特定的系数可以求解($m\neq0$无法求解)

        \subsection{谐振子势能突变}
            应该先将势场化为标准的谐振子形式$ V(x) = \frac{1}{2} \mu \omega ^{2} x^{2} $,变$k$实际上是变$\omega$
            $$ P = \abs{\intff \psiiii{0}{*}{\omega_{2},x} \psiii{0}{\omega_{1},x} dx }^{2} $$
            
            实际上和上一题有异曲同工之秒,总是拿目标基态和初始基态做内积就行了
            此题的不同点在于求平均能量,需要用到粒子现在处的态(波函数),突然变化的势场会改变处于当前态的概率,但波函数还来不及变化,使用初始波函数即可
            3
            但哈密顿量$H$(系数变了)发生了变化,不含时所以求解$t=0$时刻的能量平均值即可,
            需要带入新的哈密顿量,积分过程中和原积分进行比较(动能没变,势能变化)
            
            一个重要结论在$n=0,1$时动能和势能的期望值相等(格里菲斯$P_{33-2.11(c)}$)
            $$ <T> \quad = \quad  <V> $$

            建议记谐振子波函数的形式
            $$ \psiii{n}{\xi} = N_{n} H_{n}(\xi) e^{-\frac{\xi^{2}}{2}} $$
            $$ N_{n} = \sqrt{\frac{\alpha}{\sqrt{\pi}2^{n} n!}}  \quad \alpha = \sqrt{\frac{\mu \omega}{\hbar}} \quad \xi = \alpha x $$
            $$ \psiii{0}{\xi} = \sqrt{\frac{\alpha}{\sqrt{\pi}}} e^{-\frac{\xi^{2}}{2}} $$
            $$ \psiii{1}{\xi} = \sqrt{\frac{\alpha}{\sqrt{\pi}}} \sqrt{2} \xi e^{-\frac{\xi^{2}}{2}} \quad (H_{1}(\xi) = 2\xi) $$
            
        \subsection{1.19的演化问题}
            考虑这种演化某时长后回到某态,不再求概率,而是求$T$的某个些取值满足恒等式子,最主要的还是前两行的理解,第一行为$k \to 2k$任意含时波函数的表示,第二行
            表示此时$t = 0$的初始波函数其实为原来的基态$\psiii{0}{\omega_{1},x}$
            
            明确所需论证的是当时间为多大的$T$后,其波函数一定变为$\psiii{0}{\omega_{1},x}$(此时不仅回到基态同时$\omega_{2} \to \omega_{1}$)

            需要知道谐振子的波函数宇称为$(-1)^{n}$

        \subsection{有限区间深势阱的基态概率问题}
            由于积分区间不再是无限的,所以我们需要明确积分区间是初始波函数所在的区间
            
            关于函数的平移缩放问题
            \begin{itemize}
                \item 平移:是仅仅对$x$作加减,势场和波函数一同移动的方向满足左加右减去
                \item 伸缩:是仅仅对$x$(任何$x$加减了常数都需要拆开再变)前的系数变化,满足放大则系数缩小,反之亦然
                \item 注意如果是先平移再伸缩需要拆开括号,伸缩在平移相对不容易出错
            \end{itemize}

            归一化系系数通常是(L是整个势阱的宽度)
            $$ \frac{1}{2} L A^{2}  = 1 $$

        \subsection{有限的深势阱的移动问题}
            此题和上一题的区别在于是压缩而不是膨胀,粒子将会收到外力作用
            
            第一小问主要在于缓慢一词,粒子的状态并不发生变化
            
            第二小问在于突然一词,你无法判断每个微时刻的粒子受力情况

        \subsection{深势阱粒子的作用力问题}
            主要存在一个公式,即平均作用力做功等于基态能量的改变量,需要让能量对宽度$a$做微分(将$a$看作一个可微的变量)

            $$ F\triangle{a} = -(\frac{dE}{da}) \triangle{a} \lra F = -\frac{dE}{da} $$


        \subsection{无限深势阱的叠加态粒子}
            题中所给的波函数一定要分解为深势阱解的叠加,这样才能知道是哪个几个能量对应的本征态的叠加,乘上相因子时其能量也可解

            第二问最好用海森堡绘景的运动方程来说明

            $$ \frac{dH}{dt}  = \frac{1}{i\hbar} [H,H] = 0 $$
            
            所以 $ t = t_{0} $时的能量不变

            第三问的积分中并不是所有的相因子可以抵消,仔细计算,带入深势阱的波函时记得带入归一化系数($\sqrt{\frac{2}{a}}$)
            

        \subsection{无限深势阱的叠加态粒子2}
            第一问记得归一化波函数$\psi(x,t)$以求出$A$
            
            第二问的积分依旧难算,需要多算

            第三问不确定度的表达式为
            $$ \triangle{p} = \sqrt{<\vu*{p}^{2}> - <\vu*{p}>^{2} } $$

            记忆技巧: 平方拔 减 拔平方

            其中$\vu*{p}^{2} = 2\mu<E>$,即计算动量平方的期望需要联系上能量不用再带入计算

        \subsection{已知波函数的平均值求未知波函数平均值}
            求动量的平均值时间因子无法抵消
        
        \subsection{表象问题}

            \begin{formal}

                表象问题的公式总结:

                $$ \braket{x}{p} = \psiii{p}{x} = \frac{1}{(2 \pi \hbar)^{\frac{n}{2}}} e^{\frac{ipx}{\hbar}} $$ 
                $$ \braket{p}{x} = \psiii{x}{p} = \frac{1}{(2 \pi \hbar)^{\frac{n}{2}}} e^{\frac{-ipx}{\hbar}} $$ 
                $$ \int dx' \ket{x'} \bra{x'} = I $$ 
                $$ \int dp' \ket{p'} \bra{p'} = I $$ 
                $$ \psi(p) = \braket{p}{\psi} = \int dx' \braket{p}{x'} \braket{x'}{\psi} = \frac{1}{(2 \pi \hbar)^{\frac{n}{2}}} \intff \psi(x) e^{\frac{-ipx}{i\hbar}} dx $$
                $$ \psi(x) = \braket{x}{\psi} = \int dp' \braket{x}{p'} \braket{p'}{\psi} = \frac{1}{(2 \pi \hbar)^{\frac{n}{2}}} \intff \psi(p) e^{\frac{ipx}{i\hbar}}  dp $$
            \end{formal}
                

        \subsection{动量波函数}
            积分可以化简为
            $$ Q \int_{0}^{n \pi} \sin{u} e^{-ku} du $$
            $$ Q = \frac{a}{n\pi} \sqrt{\frac{1}{\pi \hbar a}} \quad u = \frac{n \pi \hbar}{a} \frac{x}{\hbar}  = \frac{n \pi}{a} x $$
            
            使用表格分部积分法即可,记得平方的时候,要取复共轭
           


        \subsection{深势阱的壁崩溃问题与动量波函数}
            求概率不用考虑时间因子
            
            $ p \sim p+dp $ 之间的概率为 $ \abs{\psi(p)}^{2} dp $      

            波函时的表示式子要求是$\psi(x,t)$
        
        \subsection{一维无限深势阱}
            第二问把三角函数的括号内展开,讨论$n$在级数与偶数下的函数形式,根据函数的奇偶性直接得出$ < x > = 0 $

            第三问积分使用分部积分,同时有了$\abs{c_{n}}^{2}$,可以直接通过求和获得平均能量,需要使用到一个求和公式
            $$ \bar{E} = \sum_{n} \abs{c_{n}}^{2} E_{n} \quad \sum_{n=1,3,5} \frac{1}{n^{4}} = \frac{\pi^{4}}{96} $$

            也可以使用哈密顿量求解
            $$ H = - \frac{\hbar^{2}}{2\mu} \frac{d^2}{dx^{2}} $$
            $$ \bar{E} = \int \psi^{*}(x,0) H \psi(x,0) dx $$

        \subsection{算符的本征值问题}
            需要明确以下几点
            \begin{itemize}
                \item 对易子本身就是一个算符
                \item 乘上一个新的算符只能选择左乘或者右乘
                \item 一个算符不同本征值对应不同本征函数(一般而言)
            \end{itemize}
            
            所以求证$ p_{0}+\hbar c $为其本征值,需要利用第一个小问的对易子,但是相应的本征函数不一样

        \subsection{二维谐振子耦合}
            二维谐振子的耦合有以下特点
            \begin{itemize}
                \item 哈密顿量为相加    
                \item 波函数为乘积
                \item 能量为相加
            \end{itemize}

            \begin{proof}
                \pfindent

                \begin{align}
                    H_{x} \psi(x) &= E_{x} \psi(x)  \tag{1} \\
                    H_{y} \psi(y) &= E_{y} \psi(y)  \tag{2}
                \end{align}
                
                方程(1) 乘以$\psi(y)$,方程(2) 乘以$\psi(x)$得到
                $$ (H_{x}+H_{y}) \psi(x)\psi(y) = (E_{x}+E_{y}) \psi(x)\psi(y) $$
            \end{proof}

            第一小问$N$的取值为$0,1,3\cdots,N$,所以共计$N+1$个
            
            第二小问$ n_{y} = \frac{ N - n_{x}}{2} $,必须满足$n_{y}$的取值是偶数,而$n_{x}$的取值范围为$0,1,2\cdots N$,枚举法即可

        \subsection{二维势场谐振子含交叉项}
            此题的计算方法具有极强的技巧性需要背住
            
            计算能量的本征值要先写出哈密顿量,并往标准形式上靠
            $$ \xi = \frac{1}{\sqrt{2}} (x+y) \quad \eta = \frac{1}{\sqrt{2}} (x -y)  $$
            $$ \xi^{2} + \eta^{2} = x^{2} + y^{2} \quad xy = \frac{1}{2} (\xi^{2} - \eta^{2}) \quad \pdv[2]{x} + \pdv[2]{y} = \pdv[2]{\xi} + \pdv[2]{\eta}$$
            
            记得最后变成原变量

        \subsection{两个等质一维谐振子耦合}
            变量代换同上,需要凑两次平方

        \subsection{三维谐振子}
            第一小问有第三个变量代换

            第二小问主要考虑$z<-c$时势能的情况为$\infty$,波函数必须在边界上为0,所以有$ \psi(\xi,\eta,0) = 0 $,因此$n_{3}$只能取奇数项
        
        \subsection{\texorpdfstring{$\delta$}{}势d的透射与反射问题}
            第一小问的一般表达式就是指通解(舍弃掉$x>0$部分向左传播的波)

            第三小问,透射率$T$与反射率$R$,参数带虚数指标$i$时,要取复共轭来计算  

            $$ T = \frac{\abs{F}^{2}}{\abs{A}^{2}} \quad R = \frac{\abs{B}^{2}}{\abs{A}^{2}} $$

            第四小问,一个是经典力学认为无法穿过的势垒,但实际上可以穿过的隧道效应;以及百分之百能穿的过的势阱.

        \subsection{阶跃势的透射与反射问题}
            这种题通常为了计算方便,入射系数通常取1

            第一小问由于零点两侧势能状况并不一样,即波数不一样所以需要通过概率流密度计算

            $$ j_{x} = -\frac{i\hbar}{2\mu} (\psi^{*} \pdv{}{x} \psi - \psi \pdv{}{x} \psi^{*} ) $$
            $$ T = \abs{\frac{j_{R}}{j_{I}}} \quad R = \abs{\frac{j_{T}}{j_{I}}} $$

            第二小问由于$x>0$的部分是势垒,只需要考虑波函数有界,舍去$e^{\beta x}$,其他正常算,$T$和$R$此时也满足波数一样时的等式即
            $$ T + R = 1  $$
            
            但是透射参数$T \neq \abs{F}^{2}$(解的形式不一样),由于$\psi_{2}$是实函数,所以其透射系数必为0
            
            
        \subsection{阶跃势的透射与反射问题2}
            此题中$ E = 1000ev \quad V = 750ev $,初始个数$ N_{0} = 1800 $,透射个数 $ N = N_{0} T $

        \subsection{矩形势垒的透射与反射问题}
            计算量较大,在处理$ \alpha \to 0 $时用到无穷小代换$ \sin{\alpha} = \alpha $

        \subsection{势阱的透射与散射问题}
            类似解法,同样实函数的透射系数为0
        
        \subsection{复合势:台阶势与\texorpdfstring{$\delta$}{}势的透射与散射问题}
            类似解法,解法类似,仅仅是波函数的一阶导数在0点跃进

        
        \subsection{粒子吸收模型:虚势}
            虚势场描述粒子的吸收,只是一个实用的模型,不是量子力学的理论,因为粒子在虚势场的哈密顿量不是厄米算符,它同量子力学的基本原理不符合

            这个虚势场模型描述为下

            $$ V(x) =
            \begin{cases}
            0, & \mbox{if} \quad x \mbox{ < 0} \\
            -iV , & \mbox{if} \quad x \mbox{ > 0}
            \end{cases}  $$

            前期解法基本一致,后面需要根据$ V << E$ 将 $ k $ 用$ k_{0} $表示,带入$A$,$B$取近似值.

            吸收系数的定义,单位路程上流密度的减少$ -\frac{dj}{dx} $,相对值$ \frac{1}{j} $
            $$ M = \abs{ - \frac{1}{j} \frac{dj}{dx}} $$

        \subsection{非原点的\texorpdfstring{$\delta$}{}势}

            遇到指数方程不要急,此题和三角函数方程组解法很类似
            \begin{align*}
                \frac{e^{kx} + e^{-kx}}{e^{kx} - e^{-kx}} &= 1 - \frac{2e^{-kx}}{e^{kx} - e^{-kx}} \\
                                                          &= 1 - \frac{2}{e^{2kx} - 1}
            \end{align*}
            
            将超越方程构造成过原点的直线和某一指数函数的交点问题,最后满足通过原点的直线的斜率小于另一侧指数函数的斜率$1$

        \subsection{无限深势阱的叠加态粒子3}
            算归一化系数别去积分,而是两个态的概率为1就行

            动量平均值的计算量依旧大,通常将能量差值设为

            $$ \frac{E_{2} - E_{1}}{\hbar} = w $$    
        
            最后再带入具体的

            一般的这类相似的题都有
            $$ \int \psi_{1}^{*} \vu*{p} \psi_{1} = 0 \quad \int \psi_{2}^{*} \vu*{p} \psi_{2} = 0 $$

        \subsection{海森堡绘景}
            这类题记得需要算两次微分,第一次得到的是一个微分方程组,再让方程对时间$t$求微分,并联立两个方程求解
        
            初始条件$ \vu*{x}(0) = x \quad \vu*{p}(0) = p $

        \subsection{海森堡绘景2}
            \begin{thm}\label{thm:1.3}
                \thmindent

                维里(位力)定理:动能平均值是势能的平均值的$\frac{v}{2}$倍,其中$v$表示势能函数是关于$x$的$v$次方程
                $$ \bar{T} = \frac{v}{2} \bar{V} $$ 

            \end{thm}

            第一小问在本题中$ v = -2 $,因此$ E = T + V = 0 $,不满足在该势场下$E<0$的条件

            在一维谐振子中 $ v = 2 $因此有结论$ \bar{T} = \bar{V} $

            第二小问对算符求时间的倒数时,如果含有时间项需要加上一个对时间的偏微分
            $$ \dv{\vu*{Q}(t)}{x} = \pdv{\vu*{Q}(t)}{t} + \frac{1}{i \hbar } [\vu*{Q}(t),\vu*{H}]$$

        \subsection{量子化}
            \begin{formal}
                量子化:

                量子化的一个含义是,在经典力学中取连续值的力学量,到量子力学中变成取分立值的现象,其原因是在经典力学中的力学量$F(x_{i},p_{i})$
                到了量子力学中变成了厄米算符$\vu*{F}(\vu*{x}_{i},\vu*{p}_{i})$,他们满足一些对易关系(略)

                正是这些对易关系是的一些由$\vu*{x}_{i}$与$\vu*{p}_{i}$组成的力学量算符的本征值取分立值.

                根据经典力学的哈密顿正则运动方程,带入对易关系,就得到海森堡运动方程,,这些对易关系又叫做正则量子化条件

            \end{formal}
    

    \section{力学量算符}
        \begin{formal}
            总结算符容易忘记的知识点   
            
            常见对易关系
            $$ \comm{x_{i}}{\vu*{p}_{j}} = i \hbar \delta_{ij} \quad \comm{\vu*{p}_{i}}{\vu*{p}_{j}} = 0 $$
            
            $$\comm{\vu*{L}_{i}}{\vu*{L}_{j}} = i\hbar \vu*{L}_{k} \varepsilon_{ijk}  
            \quad \comm{\vu*{L^{2}}}{\vu*{L}_{i}} = 0 \quad \comm{x}{f(\vu*{p})} = i \hbar \pdv{f(\vu*{p})}{\vu*{p}}
            $$

            $$  \comm{\vu*{L}_{i}}{\vu*{x}_{j}} = i \hbar \vu*{x}_{k} \quad \comm{\vu*{L}_{i}}{\vu*{p}_{j}} = i \hbar \vu*{p}_{k}
            $$

            $$\vu*{L}_{\pm} = \vu*{L}_{x} \pm  i \vu*{L}_{y} \quad \comm{\vu*{L}_{z}}{\vu*{{L}}_{\pm}} = \pm \hbar \vu*{L}_{\pm} \quad \comm{\vu*{L}^{2}}{\vu*{{L}}_{\pm}} = 0  $$

            角动量升降算符
            $$ \vu*{L}_{\pm} = \vu*{L}_{x} \pm  i \vu*{L}_{y} \quad \comm{\vu*{L}_{z}}{\vu*{{L}}_{\pm}} = \pm \hbar \vu*{L}_{\pm} $$
            $$ \vu*{L}_{x}  = \frac{1}{2} (\vu*{L}_{+} + \vu*{L}_{-}) \quad \vu*{L}_{y} = \frac{1}{2i} (\vu*{L}_{+} - \vu*{L}_{-}) \quad \vu*{L}^{2} = \vu*{L}_{-} \vu*{L}_{+} + \vu*{L}_{z}^{2} + \hbar \vu*{L}_{z} $$
            $$ \vu*{L}^{2} Y_{lm} = l(l+1)\hbar^{2} Y_{lm} \quad \vu*{L}_{z} = m\hbar Y_{lm}$$
            $$ \vu*{L}_{\pm} Y_{lm} = \sqrt{l(l+1) - m(m \pm 1)}  \hbar Y_{lm \pm 1} $$
            $$ \ev{\vu*{L}_{x}}{lm} = 0 \quad \ev{\vu*{L}_{y}}{lm} = 0  $$

            矩阵量子力学
            $$ \sum \ket{n}\bra{n} = 1 $$
            $$ \vu*{\rho} = \ket{\psi} \bra{\psi} \quad \vu*{\rho} =  \vu*{\rho}^{2} \quad \lambda = 0,1$$
            $$ \vu*{\rho} $$
            $$ tr(\vu*{A}) = \sum_{n} \mel{n}{\vu*{A}}{n} $$
            $$ \vu*{F}_{nm} = \mel{n}{\vu*{F}}{m} $$
                
            
            
            算符公式
            $$ F(\vu*{A}) = \sum_{n=0}^{\infty} \frac{ F^{(n)}(0) }{n!} \vu*{A}^{n} \qquad F^{(n)}(0) = \eval{\dv[n]{F(\vu*{A})}{\vu*{A}} }_{\vu*{A}=0}  $$
            $$ e^{i \alpha \vu*{A}}  =  \cos{\alpha} + i \sin{\alpha}\vu*{A} $$

            $\qquad$Baker-Hausdorff算符等式
            $$ 
            e^{\vu*{A}} \vu*{B} e^{\vu*{-A}} = \vu*{B} + \comm{\vu*{A}}{B} + \frac{1}{2!} \comm{\vu*{A}}{\comm{\vu*{A}}{\vu*{B}}} + 
            \frac{1}{3!} \comm{\vu*{A}}{\comm{\vu*{A}}{\comm{\vu*{A}}{\vu*{B}}}} \cdots 
            $$


            $\qquad$如果一个算符写在了幂指数上例如$e^{\lambda \vu*{A}}$,那么它和别的算符的对易关系可以直接看作算符$\vu*{A}$来使用,比如
            $$ \comm{\vu*{L}^{2}}{\vu*{L}_{x}} = 0 \quad \comm{\vu*{L^{2}}}{e^{\lambda \vu*{L}_{x}}} = 0 $$
        
            
            测不关系
            $$ \triangle{A} \triangle{B} \geq \frac{1}{2} \abs{\overline{\comm{A}{B}}} \quad \triangle{x} \triangle{p} \geq \frac{\hbar}{2}   $$
            $$ \triangle{A} = \sqrt{<A^{2}> - <A>^{2}} $$
            
            nabla算子运算规则
            $$ \nabla \vdot (\va{r} \frac{\psi}{r})  =  (\nabla \vdot \va{r}) \frac{\psi}{r} + \va{r} \vdot \nabla(\frac{\psi}{r}) $$
            $$ \nabla (\frac{\psi}{r}) = (\nabla \frac{1}{r})\psi + \frac{1}{r} \nabla \psi $$
            
            $\qquad$其中$ \nabla \vdot \va{r} = 3 \quad \nabla \dfrac{1}{r} = - \dfrac{\va{r}}{r^{3}} $

            $\qquad$在球坐标中
            $$ \nabla =  \pdv{r} \vb{e_{r}} +  \frac{1}{r} \pdv{\theta} \vb{e_{\theta}} + \ \frac{1}{r \sin{\theta}} \pdv{\varphi} \vb{e_{\varphi}} \quad \va{r} = r \vb{e_{r}}  $$         

            力学量完全集:
            
            \indent 它们是一组线性无关的相互对易的力学量,它们的共同本征函数全体集合可以用来表示粒子的运动状态.在力学量完全集中,力学量的个数为粒子运动
            的维数.例如对于在三维中心力场中运动的粒子,力学量完全集可以是$(x,y,z)$或$(\vu*{p}_{x} , \vu*{p_{y}} , \vu*{p}_{z})$ 或者$ (\vu*{L}^{2} , \vu*{L}_{z} , \vu*{H}) $ 如果考虑自旋,还应增加力学量$\vu*{S_{z}}$

            $ F - H $ 定理:
             $$ \pdv{E_{n}}{\lambda} = \overline{\qty(\pdv{\vu*{H}}{\lambda})_{n}} $$

            泰勒级数
            $$ \cos{x} = \sum_{n=0}^{\infty} \frac{(-1)^{n}}{(2n)!} x^{2n} = \wolfram{Series[Cos[x],{x,0,5}]} $$
            $$ \sin{x} = \sum_{n=0}^{\infty} \frac{(-1)^{n}}{(2n+1)!} x^{2n+1} = \wolfram{Series[Sin[x],{x,0,5}]} $$
            $$ e^{x} = \sum_{n=0}^{\infty} \frac{1}{n!} x^{n} = \wolfram{Series[Exp[x],{x,0,5}]} $$
            
        \end{formal}

        \subsection{空间反演算符和动量算符的厄米证明}
            换元时记得变积分上下限,负号可以收进去,同时改变积分上下限
        \subsection{球坐标的动量算符}
            一些运算规则
            $$ \nabla \vdot (\va{r} \frac{\psi}{r})  =  (\nabla \vdot \va{r}) \frac{\psi}{r} + \va{r} \vdot \nabla(\frac{\psi}{r}) $$
            $$ \nabla (\frac{\psi}{r}) = (\nabla \frac{1}{r})\psi + \frac{1}{r} \nabla \psi $$
            
            其中$ \nabla \vdot \va{r} = 3 \quad \nabla \dfrac{1}{r} = - \dfrac{\va{r}}{r^{3}} $

            在球坐标中
            $$ \nabla =  \pdv{r} \vb{e_{r}} +  \frac{1}{r} \pdv{\theta} \vb{e_{\theta}} + \ \frac{1}{r \sin{\theta}} \pdv{\varphi} \vb{e_{\varphi}} \quad \va{r} = r \vb{e_{r}}  $$         
    
        \subsection{算符函数问题}
            结论需要背住(总结部分已经有了)
        \subsection{算符欧拉公式}
            结论需要背住(总结部分已经有了)
            \begin{formal}
                两个三角函数的泰勒级数
                $$ \cos{x} = \sum_{n=0}^{\infty} \frac{(-1)^{n}}{(2n)!} x^{2n} = \wolfram{Series[Cos[x],{x,0,5}]} $$
                $$ \sin{x} = \sum_{n=0}^{\infty} \frac{(-1)^{n}}{(2n+1)!} x^{2n+1} = \wolfram{Series[Sin[x],{x,0,5}]} $$
            \end{formal}
            
        
        \subsection{算符的久期方程问题}
            要求证明某态也为某算符本征矢时,直接算符作用到态上根据条件化简到久期方程的形式,本征值是会变化的
        \subsection{对易关系求解问题}
            第一小问当已知哈密顿量时,$\vu*{p}$是可以用$\comm{x}{\vu*{H}}$表示出来的,带入后再把对易子拆开这样就能利用哈密顿量的本征值方程,注意$ \bra{n} \vu*{H} = \bra{n} E_{n} $

            求系数的时候注意最好更改能量的减法为$E_{n} - E_{m}$
            
            第二小问与第三小问需要用到单位算符
            $$ \sum_{n} \ket{n} \bra{n} = 1 $$

            第三小问化简后发现里面包含$xp$,因此将能量项提个负号再凑一个$px$

        \subsection{能量表象的算符证明}
            所谓能量表象即,能量本征方程形式
                $$ \vu*{H} \ket{n} = E_{n} \ket{n} $$

            方程中$E_{n}$为$n$个能量本征值,其本征态为$\ket{n}$

            所以将能量系数乘到算符里并作用在态上,就得到$\vu*{H}$

            分别放在第一个矩阵元和第二个矩阵元就可以得到两种形式
        \subsection{能量表象的算符证明}
            通常一个算符重复出现在一个内积里,则用单位算符隔开,如果还重复算符之间包含哈密顿算符,一般插入在哈密顿算符后面保证能用能量本征方程
            
            例如下面几个情况
            

            $\vu*{F}\vu*{{F^{\dagger}}}\vu*{H}$插入在算符之间; $\vu*{F}\vu*{H} \vu*{F^{\dagger}}$ 插入在哈密顿算符后面;特别的$\vu*{H} \vu*{F} \vu*{F^{\dagger}}$,可以直接等价$ \bra{k} \vu*{H} = \bra{k} E_{k}$并提出能量
        
        \subsection{角动量本征函数和本征值}
            $\vu*{L^{2}}$的本征函数是球谐函数$Y_{lm}(\theta,\varphi)$,本征值是$l(l+1)\hbar^{2}$,算符$\vu*{L_{z}}$的本征值是$m\hbar$
            
            $l$为角量子数取值范围是$0,1,2,3,4\cdots$,$m$是磁量子数,取值范围是$0,\pm1,\pm2,\pm3\cdots$(简并度为$(2l+1)$)

            做此题不需要把升降算符展开,直接算出对易关系就行了,算对易关系用对易算子的运算法则,建议记住$\vu*{L_{z}}$与升降算符的对易式子

            $$\vu*{L}_{\pm} = \vu*{L}_{x} \pm  i \vu*{L}_{y} \quad \comm{\vu*{L}_{z}}{\vu*{{L}}_{\pm}} = \pm \hbar \vu*{L}_{\pm} $$

            做第二小问的时候可以先把本征值假设出来,通过取符共轭做内积的方式消去本征值旁边的波函数

        \subsection{算符等式}
            \begin{formal}
                证明目标非常像一个泰勒展开

                Baker-Hausdorff算符等式
                $$ e^{\vu*{A}} \vu*{B} e^{\vu*{-A}} = \vu*{B} + \comm{\vu*{A}}{B} + \frac{1}{2!} \comm{\vu*{A}}{\comm{\vu*{A}}{\vu*{B}}} + \frac{1}{3!} \comm{\vu*{A}}{\comm{\vu*{A}}{\comm{\vu*{A}}{\vu*{B}}}} \cdots $$

                其证明过程是一个常见的方法就是加入参数$\lambda$ 

                $$ F(\lambda) = e^{\lambda \vu*{A}} \vu*{B} e^{-\lambda \vu*{A}} $$

                同一个算符与其一个算符函数的位置可以互相交换一下(因此$-\vu*{A}$可以换到最后面去)
            \end{formal}

        \subsection{算符等式2}
            第一小问回顾对易子的公式
            $$ \comm{\vu*{A}}{\vu*{B}\vu*{C}} = \vu*{B} \comm{\vu*{A}}{\vu*{C}} + \comm{\vu*{A}}{\vu*{B}}\vu*{C} $$
        
            然后利用题目条件的已知对易关系即可

            第二小问构造算符函数
            $$ F(\lambda) = e^{\vu*{A} \lambda} e^{\vu*{B} \lambda} $$

            之后使用Baker-Hausdorff算符等式时,根据题目的对易关系,有非常多项为0,因此再积分即可
        
        \subsection{角动量算符的证明题}
            Baker-Hausdorff算符等式经典用法

        \subsection{算符等式3}
            需要背住幂指数函数的展开式子
            \begin{formal}
                $$ e^{x} = \sum_{n=0}^{\infty} \frac{1}{n!} x^{n} = \wolfram{Series[Exp[x],{x,0,5}]} $$
            \end{formal}
        
        \subsection{角动量算符的本征值问题}
            要求背诵角动量升降算符的形式与本征值
            \begin{formal}
                角动量升降算符,它们都是厄米算符,可以选择作用在左边或者右边

                $$ \vu*{L}_{\pm} = \vu*{L}_{x} \pm  i \vu*{L}_{y} \quad \comm{\vu*{L}_{z}}{\vu*{{L}}_{\pm}} = \pm \hbar \vu*{L}_{\pm} $$
                $$ \vu*{L}_{x}  = \frac{1}{2} (\vu*{L}_{+} + \vu*{L}_{-}) \quad \vu*{L}_{y} = \frac{1}{2i} (\vu*{L}_{+} - \vu*{L}_{-}) \quad \vu*{L}^{2} = \vu*{L}_{-} \vu*{L}_{+} + \vu*{L}_{z}^{2} + \hbar \vu*{L}_{z} $$
                $$ \vu*{L}^{2} \ket{lm} = l(l+1)\hbar^{2} Y_{lm} \quad \vu*{L}_{z} = m\hbar \ket{lm}$$
                $$ \vu*{L}_{\pm} Y_{lm} = \sqrt{l(l+1) - m(m \pm 1)}  \hbar \ket{l \pm m} $$
                $$ \ev{\vu*{L}_{x}}{lm} = 0 \quad \ev{\vu*{L}_{y}}{lm} = 0  $$
            \end{formal}

            在求$ \overline{\vu*{L}_{y}^{2}} $时,将其中一个$\vu*{L}_{y}$作用在复共轭波函数上,提出系数时变为$- \frac{1}{2i}$,因此最前面的系数为$\frac{1}{4}$
        
        \subsection{角动量升降算符问题}
            同题目2.9(1)
        
        \subsection{未知算符和角动量的对易关系}
            这类题通常给出一个未知算符和已知算符的对易关系,由证明式子可知,在经过$\vu*{V}$算符作用后整体升了一个态变成$\ket{j+1 ,j+1}$,换句话说,如果经过算符$\vu*{L}^{2}$与算符$\vu*{L}_{z}$作用后
            能给出所有的$j$都升了$1$那么意味着算符$\vu*{V}$确实让态升了,得证

            在证明第二个等式的时候需要把$\vu*{L}^{2}$算符用升降算符和$\vu*{L}_{z}$去替代,同时由于$m=l$,所以升降算符的本征值有很多项为0
        
        \subsection{角动量算符的本征值问题2}
            第三小问注意$ l = m $, 因此$\vu*{L}_{+} \psi_{kll} = 0 $
        \subsection{连续谱问题}
            其本征函数构成正交完备系,因此任何函数都可以是他们的组合包括势函数$V(x)$,待求解的方程之中唯一未知的就是$V(x)$,所以需要将$V(x)$使用完备基表示
            
            同时由于$V(x)$仅包含$x$,因此需要将$\omega$进行积分

            左乘$u^{*}(x,\omega')$其中第二个变量是$\omega'$的原因是为了放到积分号里面(并不对$\omega'$进行积分).再进行全空间积分,得到$\delta$函数挑选出$C(\omega)(\omega-\omega_{0})$

        \subsection{算符的泰勒级数应用}
            使用幂级数展开$e^{\frac{-ia\vu*{p}_{x}}{\hbar}}$并带入算符$\vu*{p}_{x} = - i \hbar \dv{x}$,最后得到$\psi(x-a)$在$0$点的泰勒展开得证

            第二小问直接去转置共轭就可以了,不用再展开了
            \begin{formal}
                幺正算符

                $$  \vu*{U}^{\dagger} = \vu*{U}^{-1} = (\vu*{U}^{T})^{*} \llra \vu*{U}^{\dagger} \vu*{U} = \vu*{U} \vu*{U}^{\dagger}  = \vu*{I} $$
            \end{formal}

        \subsection{谐振子的均值问题}
            多个方法都要会
            \begin{formal}

                求解$\vu*{x}$和$\vu*{p}$的平均值
                \begin{itemize}
                    \item   谐振子的波函数的对称性(要么为奇函数要么为偶函数)
                    \item   递推方式 
                    $$  
                    x \psi_{n} = \frac{1}{\alpha}(\sqrt{\frac{n}{2}}\psi_{n-1} + \sqrt{\frac{n+1}{2}} \psi_{n+1})  
                    \quad \dv{\psi_{n}}{x} = \alpha (\sqrt{\frac{n}{2}}\psi_{n-1} - \sqrt{\frac{n+1}{2}} \psi_{n+1}) 
                    $$
                    \item   对易方式:将$\vu*{x}$使用$\comm{\vu*{p}}{\vu*{H}}$的对易关系来表示并带入,将哈密顿量分别左作用和右作用
                \end{itemize}

                求解$\vu*{x}^{2}$与$\vu*{p}^{2}$的平均值
                \begin{itemize}
                    \item 使用维里定理
                    \item 使用$F-H$定理
                    $$ \pdv{E_{n}}{\omega}  = \overline{\qty(\pdv{\vu*{H}}{\omega})_{n}} $$
                \end{itemize}

            \end{formal}


        \subsection{则不准关系与演化问题}
            太难算了,跳过
        \subsection{动能概率分布问题}
            当动能为$T = \frac{\vu*{p}^{2}}{2 \mu}$,则对于$p=\pm p_{0}$时都对应同一个动能,因此动能的概率分布为动量的概率分布的两倍
            $$ F(T)dT = 2 \abs{\psi(p)}^{2} dp $$

            $F(T)$本身就是概率分布函数了,不需要再平方.求动能的平均值时,由于$T$是厄米的,所以直接$\int_{0}^{+\infty} F(T) T dT $,注意积分区间是$[0,+\infty]$,
            使用高斯积分公式时值为原来的一半.把剩下的系数等价换成$\omega$更符合能量的表示形式

        \subsection{谐振子的递推公式问题}
            注意$\dv{\psi_{n}}{x}$求导的时候有两项

            第三小问将能量表示为$E - \overline{\vu*{H}} $,并用不确定度表示出来,使用均值不等式,带入测不准关系就可以得到关于能量的不等式.

            第四小问使用维里定理和$F-H$定理秒做.

        \subsection{束缚定态下的新态平均值问题}
            \begin{formal}
                
                若粒子处于束缚定态,则必有
                
                $$ <\vu*{p}> = <\vu*{F}> = 0 \quad F = -\pdv{V(x)}{x} $$

                证明方法见习题2.38(去算$\comm{x}{\vu*{H}} \quad \comm{\vu*{p}}{\vu*{H}}$)
            \end{formal}

        \subsection{谐振子演化平均值问题}
            需要利用谐振子波函数的宇称$(-1)^{n}$,同时被积函数为奇函数的直接去掉,再利用谐振子波函数的递推公式即可.


        \subsection{一维无限深方势阱定态能量}
            直接用能量的表达式就可以了.
        \subsection{正定算符}
            第一小问证明算符的正定性看是否是模方

            第二小问厄米算符左作用形成新的态,右作用形成一个态变成内积,所以正定

            \begin{formal}
                迹的定义

                $$ tr(\vu*{A}) = \sum_{n} \mel{n}{\vu*{A}}{n} $$
            \end{formal}
        \subsection{密度矩阵}

        \begin{formal}
            密度算符
            $$ \vu*{\rho} = \ket{\psi} \bra{\psi} \quad \vu*{\rho} =  \vu*{\rho}^{2} \quad \lambda = 0,1$$
        \end{formal}
            第一小问两个内积(或者矩阵元等)都是积分或者说是一个数,是可交换的
            
            第二小问取的态需使用$\phi$作为符号,以免直接内积为1,同时需要将密度算符二次作用,这是一个常见的方法.

            第三小问直接对密度算符求偏导

            $$ \pdv{\vu*{\rho}}{t} = \pdv{(\ket{\psi} \bra{\psi})}{t} $$

            题目条件的符厄米共轭方程
            $$ -i\hbar \pdv{\bra{\psi}}{t} = \bra{\psi} \vu*{H}  $$

            注意$\bra{\psi}$和$\vu*{H}$的位置要调换
        \subsection{能量表象下可观测量的本征值}
            此题仅有两个能量本征态$\ket{1} \quad \ket{2}$,所以任何态都会是它们的叠加

            在第三小问求本征值时,可以在此能量表象下求,因此可以看作$\ket{1} = \mqty(1 \\ 0) \quad \ket{2} = \mqty(0 \\ 1)$,所以算符$\vu*{R}$在此表象下的四个矩阵元可以算出

        \subsection{两套本征态的问题}
            第一小问记住,测量谁就用谁的本征态去展开此时的态,得到本征值的贡献并求和

            第二小问总结量子力学中的测量原理
            \begin{formal}
                
                量子力学测量问题

                在此题中初态为$\ket{\psi_{\lambda}}$经过测量$\vu*{A}$后,波函数坍缩到某$\ket{\phi_{n}}$上,但是这个态仅仅是体系受到测量仪器的作用
                而产生的一个暂态,是体系一个新的初态而已,体系要按照它原来的规律随时间演化也就是$\ket{\psi(t)}$
            \end{formal}

        \subsection{叉乘的计算}
            \begin{formal}
                计算叉乘使用行列式$ \vu*{p} \cross \vu*{L} $

                $$ \mqty| \va{i} & \va{j} & \va{k} \\ \vu*{p}_{x} & \vu*{p}_{y} & \vu*{p}_{z} \\ \vu*{L}_{x} & \vu*{L}_{y} & \vu*{L}_{z} | $$
            \end{formal}
            

        \subsection{角动量算符的升降算符}
            第二小问需要明确要证明什么,平均值在共同本征态上与磁量子数无关即$ \ev{\vu*{F}}{jm}  = \ev{\vu*{F}}{jm+1}$

            需要用到一个重要的等式
            $$ \vu*{J}_{+} \ket{jm} = \sqrt{j(j+1)-m(m+1)} \hbar \ket{jm+1} $$

            它的厄米共轭形式非常有用

            $$ \bra{jm} \vu*{J}_{-} = \sqrt{j(j+1)-m(m+1)} \hbar \bra{jm+1} $$

            这样$\vu*{J}_{-} \vu*{J}_{+}$不改变磁量子数的值,而算符$\vu*{J}_{-}$左作用和算符$\vu*{J}_{+}$右作用都改变磁量子数的值 

            同时该算符和升降算符都对易,即$ \vu*{J}_{-} \vu*{F} \vu*{J}_{+} = \vu*{F} \vu*{J}_{-} \vu*{J}_{+} $

        \subsection{力学量的对时间的二阶微分}
            直接使用海森堡运动方程更简单

        \subsection{位能平均值的证明题}
            需要用到题目2.6的结论,没做跳过

        \subsection{不确定关系与小孔}
            争议题
        \subsection{期望的展开}
            之前都是展开初态波函数,其厄米共轭也存在,因此可以用波函数展开某期望的形式.

            需要利用题目条件$ E_{1} \leq E_{2} \leq E_{3} \cdots $进行缩放

            第二小问基态为$\ket{0}$第一激发态为$\ket{1}$,第二激发态为$\ket{2}$,所以任一波函数的展开到第二激发态即可

            $$ \ket{\psi} = \braket{0}{\psi} \ket{0} + \braket{1}{\psi} \ket{1} + c_{2} \ket{2} $$

            其中因为要构造第二激发态所以$c_{2} = \braket{2}{\psi} = \braket{2}{2} = 1$

            $$   \ket{2} = \ket{\psi} - \braket{0}{\psi} \ket{0} - \braket{1}{\psi} \ket{1} $$
            
            求第二激发态等能量上限的计算有疑问
        
        \subsection{角动量算符的平均值}
            显然$\vu*{L}_{x}$与$\vu*{L}_{y}$都可以使用升降算符表示,在本征态下的平均值均为0(右作用磁量子数加一导致正交)

            第二小问使用简单的结论即$\vu*{L}_{x}^{2}$与$\vu*{L}_{y}^{2}$的平均值一样(对称性)

            $$ <\vu*{L}_{x}^{2}>  = <\vu*{L}_{y}^{2}> = < \frac{1}{2} (\vu*{L}^{2} - \vu*{L}_{z}^{2}) > $$

            同时和算符$\vu*{L}_{z}$的交叉相均为0,仅有$\vu*{L}_{x} \vu*{L}_{y}$和$\vu*{L}_{y} \vu*{L}_{x}$(使用升降算符计,计算量非常大)

        \subsection{束缚态的动量和力学量算符的平均值}
            主要是计算$\comm{x}{\vu*{H}}$表示$\vu*{p}$

            计算$\comm{\vu*{p}}{\vu*{H}}$,其中势函数式关于$x$的,其对易关系不是0而是$\comm{\vu*{p}}{V(x)} = \pdv{V(x)}{x} \comm{\vu*{p}}{x} = -i\hbar \pdv{V(x)}{x}$

        \subsection{反对易关系}
            简单题
        \subsection{力学量的测量问题}
            此题是一类问题需要掌握.
            \begin{formal}
                测量谁就用谁的本征态作展开(如果没有则先计算本征态),同时每次测量波函数都会坍缩到这个本征态上
            \end{formal}

            注意$\ket{\psi}$虽然测出来是本征值,但是不是本征态,要先假设本征态


        \subsection{力学量测量问题2}
            注意假设$\chi_{1}$和$\chi_{2}$是正交的,因此$\phi_{1}$展开系数的平方和为1是归一化的,而$\phi_{2}$展开系数的平方和不为1要先归一化

        \subsection{波函数的正交归一的方法}
            建议看方法2更有物理意义($P_{99}$)

        \subsection{能量表象的概率计算}
            前3问简单,注意在第三问算出中微子要么处于$\ket{e}$要么处于$\ket{\mu}$

            因此第四问再次回到$\ket{e}$,无非就是处于$\ket{\mu}$的概率为0(或者处于$\ket{e}$的概率为1),使用第三问的计算结果即可

        \subsection{谐振子的升降算符的证明问题}
            这些证明关系可以看看就行了,第二小问主要是计算$\comm{\vu*{a}_{-}}{\vu*{N}}$与$\comm{\vu*{a}_{+}}{\vu*{N}}$

            $$ \vu*{N} = \vu*{a}_{+} \vu*{a}_{-} \quad \vu*{H} = (\vu*{N} + \frac{1}{2}) \hbar \omega $$
        \subsection{谐振子的升降算符的证明问题2}
            一个重要的式子

            $$ \sum_{n=0}^{\infty} c_{n} \sqrt{n} \ket{n-1} = \alpha \sum_{n=0}^{\infty} c_{n} \ket{n} $$

            求和公式不是随便提出系数,这个公式说明的是相同态下的系数一样即
            $$ c_{n} \sqrt{n}  = \alpha c_{n-1} \quad  \llra \quad c_{n+1} \sqrt{n+1} = \alpha c_{n} $$

            得到系数的递推关系,并利用归一化条件求出$c_{0}$

        \subsection{谐振子升降算符的本征态}
            只做前两问即可,记得算符$\vu*{a}$作用上去时,提出求和项中的基态$\ket{0}$
        
        \subsection{谐振子升降算符的本征态2}
            第一小问和第二小问均是使用升降算符去表示哈密顿量、坐标算符、动量算符,由于正好处于相干态,所以算平均值是在这个态下计算
        
        \subsection{算符的对易关系证明}
            注意证明算符的对易关系需要一个任意的态,而不是题目给的某个本征态,对这个任意态用本征态做展开就行了
        

        

    \section{表象}
            \begin{formal}
                知识要点(1.27的总结)
                $$ \braket{x}{p} = \psiii{p}{x} = \frac{1}{(2 \pi \hbar)^{\frac{n}{2}}} e^{\frac{ipx}{\hbar}} $$ 
                $$ \braket{p}{x} = \psiii{x}{p} = \frac{1}{(2 \pi \hbar)^{\frac{n}{2}}} e^{\frac{-ipx}{\hbar}} $$ 
                $$ \int dx' \ket{x'} \bra{x'} = I $$ 
                $$ \int dp' \ket{p'} \bra{p'} = I $$ 
                $$ \psi(p) = \braket{p}{\psi} = \int dx' \braket{p}{x'} \braket{x'}{\psi} = \frac{1}{(2 \pi \hbar)^{\frac{n}{2}}} \intff \psi(x) e^{\frac{-ipx}{i\hbar}} dx $$
                $$ \psi(x) = \braket{x}{\psi} = \int dp' \braket{x}{p'} \braket{p'}{\psi} = \frac{1}{(2 \pi \hbar)^{\frac{n}{2}}} \intff \psi(p) e^{\frac{ipx}{i\hbar}}  dp $$
                $$ \delta(x) = \frac{1}{(2\pi)^{n}} \intff e^{ipx} dp $$
                $$ \delta(p) = \frac{1}{(2\pi)^{n}} \intff e^{-ipx} dx $$

                一维动量波函数满足的定态方程
                $$ \frac{p^{2}}{2\mu} \varphi(p) + \intff V_{pp'} \varphi(p') dp' = E\varphi(p) $$
                $$ V_{pp'} = \frac{1}{2\pi \hbar} \intff e^{-i(p-p')x/\hbar} V(x)dx $$
                
                如果$V(\va{r})$可以表示成$\va{r}(x,y,z)$的正幂次级数,定态方程变为
                $$ \big[\frac{p^{2}}{2\mu}+V(r=i\hbar \nabla_{p})\big]\varphi(p) = E\varphi(p) $$

                

                矩阵元
                $$ \vu*{F}_{nm} = \mel{n}{\vu*{F}}{m} = \int u^{*}_{m} \vu*{F} u_{n} d\tau  $$

                表象变换
                $$ \psi' = S^{\dagger} \psi \quad \vu*{F}' = S^{\dagger} \vu*{F} S $$

                其中$S$矩阵可以在$\vu*{Q}$表象中求出$\vu*{Q}'$的所有本征态矢



            \end{formal}
        
        \subsection{动量表象下的定态能量和波函数}
            此求解过程典型,对$p$求导的时候$p'$对其的导数为$0$,本质上它们不是同一个$p$
            
            $$ \dv{\varphi(p)}{p} = - \frac{d(p^{2}+2\mu \abs{E})}{p^{2}+2\mu \abs{E}} $$
            $$ \ln{\varphi(p)}+\ln{(p^{2}+2\mu \abs{E})} = c                           $$
            $$ \varphi(p) = \frac{A}{p^{2}+2\mu \abs{E}}                               $$

            \begin{formal}
                
                两个需要学会的积分
                \begin{itemize}
                    \item 积分1
                        $$\int \frac{1}{x^{2}+1} dx  = \arctan{x} \lra \int \frac{1}{x^{2}+k^{2}} dx = \frac{1}{k} \arctan{\frac{x}{k}} $$
                    \item 积分2
                        $$ \intff \frac{1}{(x^{2}+b^{2})^{2}} dx = \frac{1}{2} b^{-3} \pi $$
    
                    \begin{proof}
                        \pfindent
                        $$ x = b\tan{\theta} \quad dx = b \frac{1}{\cos{\theta}^{2}} d\theta \quad 1+\tan^{2}{\theta} = \frac{1}{\cos^{2}{\theta}} $$

                        积分区间变为$ (-\frac{\pi}{2},\frac{\pi}{2}) $,通过以上换元即可求得结果
                    \end{proof}

                \end{itemize}
            \end{formal}
            
        \subsection{p表象下的测不准关系验证}
            没有技巧都是积分,$\intff \frac{x^{2}}{(x^{2}+b^{2})^{2}} dx $同样适用上面的换元        

        \subsection{谐振子的p表象计算}
            $ x = i\hbar \pdv{p}$,证明见题目1.16
            最后将方程化成类似于坐标表象下的薛定谔方程,解的形式类似

        \subsection{均匀力场问题}
            $$ f(x) = -F = - \pdv{V(x)}{x} \lra V(x) = F x \lra V(p) = Fi\hbar \dv{p}$$

        \subsection{概率密度对时间的导数}
            这类题不涉及能量,因此能量写为$i\hbar \pdv{t}$,取符共轭去计算概率密度对时间的导数

        \subsection{自由粒子两种表象下的均值计算}
            主要在于第二小问,自由粒子的动量守恒,动量的平均值不随时间变化,平均位移等于平均速度$\frac{p_{0}}{\mu}$乘以时间$t$,或者带入海森堡运动方程中进行计算
        
        \subsection{中子反中子的波函数计算}
            这道题计算的关键在于要求出$t$时刻的波函数,而整个体系的哈密顿量为$\vu*{H} = \vu*{H}_{0} + \vu*{H}'$,取$\vu*{H}_{0}$为表象,其两个简并态作为基底即可

            需要注意求出能量本征值后,需要计算此时哈密顿量的本征态(并不是$\ket{n}$,$\ket{\bar{n}}$)

            需要在矩阵表达式中带入能量 
            
            $$ \mqty(-\alpha & \alpha \\ \alpha & -\alpha) \mqty(c_{1} \\ c_{2}) = 0 $$

            并且有额外归一化要求$ \abs{c_{1}}^{2} + \abs{c_{2}}^{2} = 1 $,然后将获得的本征态用$\ket{n}$,$\ket{\bar{n}}$展开
            
            求出两个本征态后,任意时刻的波函数用它们展开,记得带系数(由初始条件决定,此题的初始条件为中子态).

            然后带入能量与此时的本征态先化简再转化成$\ket{n}$,$\ket{\bar{n}}$的展开

        \subsection{矩阵形式的力学量测量问题}
            第二小问,初态题目已经给了,需要测量力学量$\vu*{A}$所以需要用它的本征态作展开,第一步就是求解它的本征值与本征态

            经过计算$t=0$时刻只能测得本征值$a$而它是简并的,所以处于这两个态其中之一或者它们的线性组合

            第三小问注意$t$时刻的波函数的时间因子只能是哈密顿量的能量,不能是其他力学量的本征值,化简后再计算每个力学量的期望值
        \subsection{表象的本征值对应的矩阵}
            \begin{formal}
                在自身表象下所有的本征值,其算符的矩阵表示是对角化的
                $$ \vu*{A} = \mqty (\lambda_{1} & 0 & 0 \\ 0 & \lambda_{2} & 0 \\ 0 & 0 & \lambda_{3}) $$
            \end{formal}
        
            再讲算符$\vu*{B}$的矩阵参数设出来,并根据题目已知条件进行求解(反对易关系以及厄米性以及平方为1)

            注意这个过程似乎和$\vu*{A}$表象没有任何关系,但实际上在利用反对易关系的时候就涉及到了$\vu*{A}$自身表象的矩阵形式,因而$\vu*{B}$的矩阵是在此表象下求得的

            第二小问由于其在$\vu*{A}$表象下$\vu*{B}$的矩阵形式已经获得,因此直接解它的本征值方程即可

            第三小问回顾表象变换
            $$ \psi' = S^{\dagger} \psi \quad \vu*{F}' = S^{\dagger} \vu*{F} S $$
            \begin{formal}
                
                求解幺正变换矩阵的方法
                \begin{itemize}
                    \item 已知变换前后的两组基矢$S_{kb} = \braket{\psi_{k}}{\phi_{b}} $
                    $$\mqty(a_{1}' \\ a_{2}' \\ a_{3}') = \mqty(S_{11} & S_{12} & S_{13} \\ S_{21} & S_{22} & S_{23} \\ S_{31} & S_{32} & S_{3})  \mqty(a_{1} \\ a_{2} \\ a_{3}) $$ 
                    \item 已知力学量$\vu*{B}$在$\vu*{A}$表象下矩阵表示,直接求解本征态失,并起来可构成幺正变换矩阵$S$
                \end{itemize}
            \end{formal}

        \subsection{角动量表象}
            第一小问$\vu*{L}_{z}$在自身表象下是对角化的,其他角动量根据矩阵元的计算方法进行计算,需要用到升降算符

            计算如$(\vu*{L}_{x})_{12}$与$(\vu*{L}_{x})_{21}$时,只需要计算一边,再去复共轭即可(它们是厄米算符)

        \subsection{角动量表象本征值本征态}
            方法简答,略

        \subsection{角动量表象下的测量问题}
            题目给的$2\hbar^{2}$得出$l=1$的结论

            完整的做题过程需要在这三个基底下$Y_{1-1},Y_{10},Y_{11}$写出$\vu*{L}_{y}$的矩阵表示,并求出本征值和本征态(前两个题已经算过)

            再由于这个量子体系必然处于这三个基底中的一个态上,则测量的概率是$\abs{\braket{\varphi_{0}^{\dagger}}{\psi}}^{2}$其中$\psi$为这三个中的一个,分别计算
            
        \subsection{直角坐标系波函数转化计算角动量}
            题目给出的是直角坐标系下的波函数,需要将其改写成球坐标系下的形式,并利用球谐函数表示它

            \begin{formal}
                涉及的知识点(背住)

                三个典型球谐函数
                $$ Y_{11} = -\sqrt{\frac{3}{8\pi}} \sin{\theta} e^{i\varphi} \quad Y_{10} = \sqrt{\frac{3}{4\pi}}\cos{\theta} \quad Y_{1-1} = \sqrt{\frac{3}{8\pi}} \sin{\theta} e^{-i\varphi}  $$

                l=1,的$(\vu*{L}^{2},\vu*{L}_{z})$表象下

                $$ 
                \vu*{L}_{x} = \frac{\hbar}{\sqrt{2}} \mqty(0 & 1 & 0 \\ 1 & 0 & 1 \\ 0 & 1 & 0) \quad 
                \vu*{L}_{y} = \frac{\hbar}{\sqrt{2}} \mqty(0 & -i & 0 \\ i & 0 & -i \\ 0 & i & 0) \quad 
                \vu*{L}_{z} = \hbar \mqty(1 & 0 & 0 \\ 0 & 0 & 0 \\ 0 & 0 & -1)
                $$

                它们三个的本征值都是$\hbar,0,-\hbar$,本征态自行计算

            \end{formal}

        \subsection{角动量表象纠缠态问题1}
            第三小问需要将给出的$\psi$用矩阵形式表达,并写出$\vu*{L}_{x}$的本征值和本征态,再做计算


        \subsection{角动量表象纠缠态问题2}
            没看懂这道题,略

        \subsection{角动量表象j=3/2}
            需要将4个球谐函数作为基底,此表象下$\vu*{j}^{2}$和$\vu*{J}_{z}$的矩阵可以根据对角化直接写出

        \subsection{角动量表象下的本征值和本征态问题}
            无论在什么表象下三个角动量分量算符的本征值取值都是$m\hbar$,所以此题最大的本征值为$\frac{3}{2}\hbar$
        
        \subsection{共同本征态问题}
            第二小问能量$E=E_{0}$是非简并的,能量$E=-E_{0}$是简并的.容易验证$\ket{1}$是本征态,单独的$\ket{2},\ket{3}$不是,而它们的线性组合仍旧是本征值$-E_{0}$的本征态,
            所以存在合适的系数使得这个混合态是$\vu*{A}$D的本征态
        
        \subsection{角动量表象下哈密顿量的表示}
            要背住三个分量在$l=1$,动量表象下的矩阵表示,带入哈密顿量即可.(换元简化形式)

        \subsection{动量表象下的哈密顿量}
            需要知道$r = i\hbar \nabla_{p} = i\hbar (\pdv{p_{x}} + \pdv{p_{y}} + \pdv{p_{z}} )$
            
            得到$ V(\vu*{r} = i\hbar \nabla_{p})$ 后需要转化回原来的坐标表象下(凑$(i\hbar)^{2}$)

            nabla算子作用后记得带各方的单位向量,转回向量$\va{r}$


    \section{三维定态问题}
            \begin{formal}
                知识点总结
                
                \begin{enumerate}
                    \item 在中心力场$V(r)$中,定态波函数$\psi(r)$可以表示为
                    $$ \psi(r) = R(r) Y_{lm}(\theta,\varphi) = \frac{u(r)}{r}Y_{lm}(\theta,\varphi) $$

                    归一化形式变为
                    $$ \int_{0}^{\infty} \abs{\psi(r)}^{2} 4\pi r^{2} dr = 4 \pi \int_{0}^{\infty}\abs{u(r)}^{2} dr = 1 $$
                    
                    其中$R(r)$满足的方程为
                    $$ \dv[2]{R(r)}{r} + \frac{2}{r} \dv{R(r)}{r} + \left\{ \frac{2\mu}{\hbar^{2}}[E-V(r)] - \frac{l(l+1)}{r^{2}} \right\}   R(r)  = 0 $$

                    $u(r)$满足方程与边界条件
                    $$ \dv[2]{u(r)}{r} + \left\{   \frac{2\mu}{\hbar^{2}}[E-V(r)] - \frac{l(l+1)}{r^{2}} \right\} u(r) = 0 \quad u(0) = 0 $$

                    \item 带有电荷$q$的粒子在电磁场中的哈密顿量算符为(朗道量子化
                    是指均匀磁场中带电粒子的回旋轨道发生的量子化。这些带电粒子能量在一系列分立的数值中取值,形成朗道能级。朗道能级是简并的,每一能级上电子的电子数量与外加磁场的强度成正比)
                    $$ \vu*{H} = \frac{1}{2\mu} [\vu*{p} - \frac{q}{c} A(r,t) ]^{2} + q \Phi(r,t) $$

                    其中$\vu*{p} = -i\hbar \nabla$,$A(r,t)$与$\Phi(r,t)$分别是电磁场的矢势和标势.

                    电磁场的矢势(矢势的取法有无穷多种)
                    $$ \va{B} = \nabla \times \va{A}  \lra  \va{A'} = \va{A} + \nabla \varphi $$

                    当磁场$ \va{B} = B \va{k} $
                    $$ \va{A} = (-By,0,0) \quad \va{A} = (-\frac{By}{2},\frac{Bx}{2},0) $$
                    
                    电磁场的标势

                    $$ \va{E} = -\nabla \varphi - \pdv{\va{A}}{t} $$
                    
                    波函数为$\psi$的粒子在电磁场中的概率流密度为

                    $$ \vb{J} = \frac{1}{2\mu} \big[ \psi^{*}(\vu*{p} - \frac{q}{c} \vb{A} )\psi + \psi(\vu*{p} - \frac{q}{c} \vb{A} )^{*} \psi^{*} \big] $$
                    $$ \vb{J} = \frac{1}{2} [\psi^{*} \vu*{v} \psi + \psi \vu*{v} \psi^{*}] $$                 
                    $$ \vu*{v} = \frac{1}{\mu} (\vu*{p} - \frac{q}{c} \vb{A}) $$
                    这里的$\vu*{v}$是粒子的速度算符

                    对比非电磁场下的概率流密度
                    $$ j_{x} = -\frac{i\hbar}{2\mu} (\psi^{*} \pdv{}{x} \psi - \psi \pdv{}{x} \psi^{*} ) $$

                    \item 在三维无限深方势阱
                    $$
                    V(x,y,z) =
                    \begin{cases}
                    0, & 0<x<a,0<y<b,0<z<c  \\
                    \infty, & \mbox{其他}
                    \end{cases}
                    $$
                    中,定态能量和定态波函数为
                    $$ E_{n_{1}n_{2}n_{3}} = \frac{\pi^{2}\hbar^{2}}{2\mu} \big(\frac{n_{1}^{2}}{a^{2}} + \frac{n_{2}^{2}}{b^{2}} + \frac{n_{3}^{2}}{c^{2}} \big)$$
                    $$ \psi_{n_{1}n_{2}n_{3}}(x,y,z) = 
                    \begin{cases}
                        \sqrt{\frac{8}{abc}} \sin{\frac{n_{1}\pi x}{a}} \sin{\frac{n_{2}\pi y}{b}} \sin{\frac{n_{3}\pi z}{c}}, & \mbox{阱内}\\
                        0, & \mbox{阱外}
                    \end{cases}
                    $$

                    \item 在三维各向异性谐振子势场
                    $$ V(x,y,z) = \frac{1}{2} \mu ( \omega_{1}^{2} x^{2} + \omega_{2}^{2} y^{2} + \omega_{3}^{2} z^{2}) $$
                    中,定态能量和定态波函数为
                    $$ E_{n_{1}n_{2}n_{3}} = (n_{1} + \frac{1}{2})\hbar \omega_{1} + (n_{2} + \frac{1}{2})\hbar \omega_{2} + (n_{3} + \frac{1}{2})\hbar \omega_{3} $$
                    
                    $$ 
                    \psi_{n_{1}n_{2}n_{3}}(x,y,z) = N_{n_{1}}N_{n_{2}}N_{n_{3}} exp[ -\frac{1}{2}(\alpha_{1}^{2}x^{2} + \alpha_{2}^{2}y^{2} + \alpha_{3}^{2}z^{2}) ] \times 
                    H_{n_{1}}(\alpha_{1}x)H_{n_{2}}(\alpha_{2}y)H_{n_{3}}(\alpha_{3}z) 
                    $$
                    
                    $$ \alpha_{i} = \sqrt{\frac{\mu \omega_{i}}{\hbar}} \quad N_{n_{i}} = \sqrt{\frac{\alpha_{i}}{\sqrt{\pi 2^{n_{i}}n_{i}!}}}  \quad n_{i} = 0,1,3,4\cdots \quad i=0,1,3,4\cdots $$

                    \item 在类氢离子势场$V(r) = -\frac{Ze^{2}}{r}$中,定态能量和定态波函数为
                    $$ E_{n} = - \frac{Z^{2} e^{2}}{2a n^{2}} $$
                    $$ \psi_{nlm}(r) = R_{nl}(r) Y_{lm}(\theta,\varphi) $$
                    $$ R_{nl}(r) = N_{nl} e^{-\frac{Zr}{na}} (\frac{2Zr}{na})^{l} F(l+1-n,2l+2,\frac{2Zr}{na}) $$
                    $$ N_{nl} = \frac{2Z^{3/2}}{a^{3/2}n^{2}(2l+1)!} \sqrt{\frac{(n+l)!}{(n-l-1)!}}  $$
                    其中$a$是波尔半径,$F(l+1-n,2l+2,\frac{2Zr}{na})$是个合流超几何函数

                    主量子数$n$(电子层),角量子数$l$(角动量轨道),磁量子数$m$(磁矩方向)
                    $$ n = 1,2,3,4\cdots \quad l = 0,1,2,3\cdots n-1 \quad m = 0,\pm1,\pm2,\pm3,\cdots,\pm l $$

                    电子层的代号($KLMNOP$),亚层轨道代号($spdf$),当说$s$波时即$l=0$

                    \item 角动量的三个分量,以及$\vu*{L}_{z}$与$\varphi$角的关系

                    $$ \vu*{L}_{x} = y p_{z} - z p_{y} \quad \vu*{L}_{y} = z p_{x} - x p_{z} \quad \vu*{L}_{z} = x p_{y} - y p_{x} \quad $$
    
                    柱坐标系下
                    $$ \vu*{L}_{z} = -i\hbar \pdv{\varphi} \quad \vu*{L}_{z}^{2} = -\hbar^{2} \pdv[2]{\varphi} $$
    
                    此时$\vu*{L}_{z} = -i\hbar \pdv{\varphi}$的本征函数为(曾书$P_{137}$)
    
                    $$ \psi_{m}(\varphi) = \frac{1}{\sqrt{2\pi}} e^{im\varphi} $$

                    \item 类氢原子的玻尔半径
                    $$ a = \dfrac{\hbar^{2}}{\mu e^{2}} $$

                    \item 保守系下的维里(位力)定理
                    $$ \overline{T} = \frac{1}{2} \overline{\va{r} \vdot \nabla V(r)} $$

                    \item 折合质量(两体问题变单体问题)
                    $$ \mu = \dfrac{m_{A}m_{B}}{m_{A}+m_{B}} $$


                \end{enumerate}
            \end{formal}

        \subsection{三维方势阱}
            由题目条件束缚态$E<0$,$\quad s$波$\lra l = 0$

        \subsection{\texorpdfstring{$\delta$}{}势阱}
            存在束缚态即至少存在一个态为束缚态,那么必然是$l=0$的基态

            同样是指数式的超越方程,构造成过原点的直线和指数函数,同1.43题

        \subsection{无限深势阱}
            当波函数在某个非0常数时需要取得0时,可以直接假设波函数的形式为$u(r) = A \sin{k(r-a)}$

            方法和习题1.6的偶宇称解有点儿类似

            注意积分过程中$\sin{2k(b-a)} = 0$

        \subsection{已经势场条件求能量和势能}
            此题不用按照书上的求法,直接让$\psi(r)r = u(r) $带入$u(r)$满足的定态方程就可以了

        \subsection{柱坐标系的矢势问题}
            \begin{formal}
                需要补充以下知识点

                $$ \va{B} = \nabla \times \va{A} $$

                $$ 
                \mqty|\va{i} & \va{j} & \va{k} \\ \pdv{x} & \pdv{y} & \pdv{z} \\ A_{x} & A_{y} & A_{z}|  = 
                (\pdv{y}A_{z} - \pdv{z}A_{y}) \va{i} + (\pdv{z}A_{x} - \pdv{x}A_{z}) \va{j} + (\pdv{x}A_{y} - \pdv{y}A_{x})\va{k} 
                $$

                当磁场为$z$轴方向时,即$\va{B} = B \va{k}$

                $$ \pdv{x}A_{y} - \pdv{y}A_{x} = 0 $$

                于是我们可以得到至少两种常见取法
                \begin{itemize}
                    \item $$ \va{A} = (-By,0,0) $$
                    \item $$ \va{A} = (-\frac{B}{2}y,\frac{B}{2}x,0) $$
                \end{itemize}

                通常无其他限制取第一种,如果被限制在了$xy$平面,比如涉及到柱坐标系,那么使用第二种

                角动量的三个分量,以及$\vu*{L}_{z}$与$\varphi$角的关系

                $$ \vu*{L}_{x} = y p_{z} - z p_{y} \quad \vu*{L}_{y} = z p_{x} - x p_{z} \quad \vu*{L}_{z} = x p_{y} - y p_{x} \quad $$

                柱坐标系下
                $$ \vu*{L}_{z} = -i\hbar \pdv{\varphi} \quad \vu*{L}_{z}^{2} = -\hbar^{2} \pdv[2]{\varphi} $$

                此时$\vu*{L}_{z} = -i\hbar \pdv{\varphi}$的本征函数为(曾书$P_{137}$)

                $$ \psi_{m}(\varphi) = \frac{1}{\sqrt{2\pi}} e^{im\varphi} $$

            \end{formal}

        \subsection{磁场下的电荷定态能量和波函数}

            取磁矢势能$\va{A} = (-By,0,0)$
            $$ \vu*{H} = \frac{1}{2\mu} [(\vu*{p}_{x}^{2} + \frac{qB}{c} y)^{2} + \vu*{p}_{y}^{2} + \vu*{p}_{z}^{2} ] $$

            这里一共有四个量$\vu*{p}_{x} \quad \vu*{p}_{y} \quad \vu*{p}_{z} \quad y $

            分别与哈密顿量计算对易子,显然只有$\vu*{p}_{x} \quad \vu*{p}_{z}$是对易的

            它们本征函数的形式是
            $$ \frac{1}{2\pi \hbar} e^{ip_{x}x/\hbar} \quad \frac{1}{2\pi \hbar} e^{ip_{z}z / \hbar} $$

            剩下的部分是关于变量$y$的函数,因此波函数可以表示为(常数归化到$\varphi(y)$里面了)
            $$ \psi(x,y,z) = e^{i(p_{x}x+p_{z}z)/\hbar} \varphi(y) $$

            带入哈密顿量的本征方程,指数项都消掉了,所以得到的是关于$y$的方程,其他变量都可以看作常数,得到一个类似谐振子的方程
            
            算出$E'$与$\varphi(y)$后记得带回去算出$\psi(x,y,z)$与$E$


        \subsection{电磁场下的电荷定态能量和波函数}
            方法和上题类似,只是凑平方更加复杂.得到饿依旧是谐振子的解

        \subsection{磁场和势场下的运动}
            \begin{formal}
                对易子的一个重要计算
                    $$ \comm{A}{(B+C)} = \comm{A}{B} + \comm{A}{C} $$
            \end{formal}
        
            需要知道的是,矢势$\va{A}_{x} \quad \va{A}_{y} \quad \va{A}_{z}$它们都是坐标$(x,y,z)$的函数,求对易的时候直接用公式
            $$ \comm{\vu*{p}_{x}}{A_{y}}  =  \comm{\vu*{p}_{x}}{x} \pdv{x} A_{y} =  - i \hbar \pdv{x} A_{y} $$

            虽然$A_{y}$是$(x,y,z)$的函数,但是$y,z$和$\vu*{p}_{x}$都是对易的,所以用这个公式没有问题,或者用试探函数法解出来的效果一样

            第二小问$\comm{P}{A} = -i\hbar \nabla \vdot A = 0$(这是三维情况下的对易公式)

            第三小问的一个知识点

            \begin{formal}
                薛定谔方程是正则量子化得到的,动量是不随时间变化的
            \end{formal}

            所以此问应该用动量表象,分离变量

        \subsection{类氢原子核电荷数突变}
            跳过

        \subsection{氢原子基态波函数的演化问题}
            自由电子的波函数
            $$ \psi_{p} = \frac{1}{(2\pi \hbar)^{\frac{3}{2 }}} e^{ip \vdot r/\hbar} $$
            动量表象下$p~p+dp$内的概率为
            $$ \abs{\varphi(p)}^{2} 4\pi p^{2} dp $$

            跳过此题

        \subsection{钢球势阱}
            第三小问主要在于积分区间的问题,因为初态波函数在$R-2R$上没有定义,所以那部分的积分是没有的,假如积分区间写为$0-2R$会发现计算出来是0

            同理,如果是计算半径缩小一半的情况,积分区间就变为$0-\frac{R}{2}$

        \subsection{能量与作用力和压强}
            \begin{formal}
                \begin{itemize}
                    \item 能量与作用力
                    \begin{enumerate}
                        \item 粒子对外做功$ F dR $等于能量的减少量$ - dE \lra F = - \dv{E}{R}$
                        \item 能量$ E = \frac{p^{2}}{2\mu} \lra dE = \frac{p}{\mu} dp = v dp $,
                        作用力$ F = \dv{p}{t}  \quad v dt = dR \lra F = \abs{\dv{E}{R}}$
                    \end{enumerate}
                    \item 能量与压强 
                    
                    在球壳形式下$ P = \dfrac{F}{4\pi R^{2}} $
                \end{itemize}
            \end{formal}

        \subsection{中心势场下的维里定理的证明}

            \begin{formal}
                证明维里定理,主要是借助算符$ \va{r} \vdot \vu*{p} $不随时间变化,带入海森堡方程
                $$ 0 = \dv{t} \overline{(\va{r} \vdot \vu*{p})} =  \frac{1}{i\hbar} \overline{\comm{\va{r} \vdot \vu*{p}}{\frac{p^{2}}{2\mu}+V(r)}} $$
            \end{formal}

            自己要会计算右边的对易,需要注意点乘也是可以展开的

            $$ 
            \comm{\va{r} \vdot \vu*{p} }{\frac{p^{2}}{2\mu}} = \va{r} \vdot \comm{\vu*{p}}{\frac{p^{2}}{2\mu}} + 
            \comm{\va{r}}{\frac{p^{2}}{2\mu}} \vdot \vu*{p} = \frac{i\hbar}{\mu}\vu*{p}^{2} 
            $$

            第二小问可以忽略涉及场论

        \subsection{类氢原子下计算\texorpdfstring{$\frac{1}{r}$}{}的平均值}
            维里定理$ E_{n} = <T> +  <V>  $

            又或者使用$F-H$定理对$e^{2}$求导(通用方法)

        \subsection{类氢原子下计算\texorpdfstring{$\frac{1}{r^{2}}$}{}的平均值}
            在利用$F-H$定理时,通常要把波动方程写成左边有效哈密顿量,右边能量
            $u(r)$满足的方程可以改写为以下形式(类氢原子)
            $$ 
            \qty[ -\frac{\hbar^{2}}{2\mu} \dv[2]{r} - \frac{Ze^{2}}{r} + 
            \frac{l(l+1)\hbar^{2}}{2\mu r^{2}} ] u_{nl}(r) = E_{n} u_{nl}(r)  \llra  \vu*{H}_{eff} u_{nl}(r) = E_{n} u_{nl}(r)
            $$

            \begin{formal}
                类氢原子的能级与一般中心势(例如球方势阱)能级不同的特点,即能级只依赖与径向量子数$n_{r}$($n_{r} = 0,1,2,3\cdots$),角量子数$l$的一种特殊组合,即只依赖于主量子数$n = n_{r} + l + 1$
            \end{formal}

            因此此题对$l$求导时,需要将能量$E_{n}$的主量子数,写成$n_{r}+l+1$

            \begin{formal}
                类氢原子的玻尔半径
                $$ a = \dfrac{\hbar^{2}}{\mu e^{2}} $$
            \end{formal}

            因此如果打算选择其他参数,则需要把玻尔半径展开,所以$l$的求导更合适

        \subsection{径向角动量}
            第一小问正如题目2.2那样,最好作用在一个任意的态上,nabla算子需要带入球坐标系下的表示

            \begin{formal}
                一个nabla算子的误区

                计算原则
                $$ \nabla \vdot (\varphi \va{f}) = (\nabla \vdot \varphi)\va{f} + \varphi(\nabla \vdot \va{f}) $$

                由此容易得出错误结果
                $$ \nabla \vdot (\frac{\va{r}}{r}) = \frac{2}{r} $$

                在量子力学中算符通常需要作用在某一个态上,所以此时并不是单纯的点乘$\frac{\va{r}}{r}$,而是点乘$\frac{\va{r}}{r}\psi$
                $$ 
                \nabla \vdot (\frac{\va{r}}{r} \psi) = \frac{\va{r}}{r} \vdot \nabla \psi +  (\nabla \vdot \frac{\va{r}}{r}) \psi \lra 
                = \frac{\va{r}}{r} \vdot \nabla + \frac{2}{r}
                $$
            \end{formal}

            第二小问需要知道拉普拉斯的球坐标系下的表示形式,以及角动量平方在球坐标系下的表示.(跳过)

            后面几问都很复杂跳过

        \subsection{基态氢原子的动量概率分布函数}
            在三维情况下
            $$ \varphi(p) = \dfrac{1}{(2\pi \hbar)^{\frac{3}{2}}} \int_{0}^{+\infty} \psi(r) e^{i p \vdot r/ \hbar} dr  \quad p \vdot r = p r \cos{\theta} $$
            $$ W(p) = \abs{\varphi(p)}^{2} 4\pi p^{2}  $$
            积分不算很好计算

        \subsection{基态氢原子检验测不振原理}
            在球对称情况下$\overline{x^{2}} = \overline{y^{2}} = \overline{z^{2}} = \frac{1}{3} \overline{r^{2}} $
            
            一旦知道势场函数和能量函数(氢原子的需要背),考虑使用维里定理得到$<T>$更利于计算$\overline{p^{2}}$

        \subsection{极坐标下的不确定关系}
            \begin{formal}
                反三角函数的导数,为直接函数的倒数

                $$ \dv{x}{y} = \dfrac{1}{\dv{y}{x}} $$

                放在三角型里面看,如$\arcsin{x}$,斜边为1,对边为$x$,角度$y$,因此导数为斜率$\dfrac{1}{\sqrt{1-x^{2}}}$
            \end{formal}

            偏导数的转化

            $$ \pdv{x} = \pdv{\varphi} \pdv{\varphi}{x} $$

        \subsection{测不准关系估算氢原子基态能量}
            基态下$l=0$,需要用到4.16题的哈密顿量

        \subsection{测不准关系估算基态能量}
            估算题可以跳过(简单看一下)
            
        \subsection{测不准关系估算基态能量2}
            估算题可以跳过(简单看一下)

        \subsection{维里定理的一般描述}
            \begin{formal}
                维里(位力)定理:质点系总动能对时间的平均值,等于作用在质点系上的位力。

                $$ \overline{T} = -\frac{1}{2} \overline{\sum{\va{r}_{i} \vdot \va{F}_{i} }} $$

                等式右边这一块叫做"位力"

                在保守系下  $ \va{F} = - \nabla V(r) = - \dv{r} V(r) $
                
                $$ \overline{T} = \frac{1}{2} \overline{\sum{\va{r}_{i} \vdot \nabla V(r) }} $$
                
            \end{formal}
            
            在本题中
            $$ \va{r} \vdot \nabla V = r \dv{V}{r} = r \dv{r}(c \ln{\frac{r}{r_{0}}}) = c  $$

            第二小问,使用$F-H$定理时,哈密顿对参数求导和记得取平均,利用之前的条件

            此题具有启发性,这告诉我们此定理不仅可以在能量已知的情况下,求解某个变量的平均(包含在哈密顿量内),也可以求解能量关于某个参数的具体形式

            $$ E_{n} = - \frac{c}{2} \ln{\mu} + C_{n} $$
            
            积分常数$C_{n}$是一定同变量$\mu$无关的数,所以对任意的$n$,能量之差都和质量无关

        \subsection{两体问题}
            \begin{formal}
                两体问题的折合质量

                当考虑两个相互作用的质点(弹簧连接的两个物体,氢原子中质子和电子的相互束缚),为了方便研究整个体系,将其中一个质点A是为静止,另一个质点B作运动,其折合质量为
                
                $$ \mu = \dfrac{m_{A}m_{B}}{m_{A}+m_{B}} $$

                在研究氢原子体系的波函数时,因为质子质量远大于电子质量,所以带入上式子可知约化质量为电子质量

                注意区分质心质量和折合质量(研究方法不同,很多问题研究质心时,质心是不动的,而约化质量是研究运动的)

                因此在这道题中类似于氢原子体系,两个质点相互束缚,不过将氢原子势能换成了其他三维球势,所以使用约化质量,
                把其中一个看成静止的,另一个围绕它运动,折合质量为$\frac{m}{2}$
            \end{formal}
 


        \subsection{中心力场下的能量本征波函数}
            事实上中心力场下的能量本征波函数就是我们熟知的
            
                $$ \psiii{nlm}{r} = R_{nl}(r) Y_{lm}(\theta,\varphi) $$

            只要求了径向积分则只有$r^{2}dr$(舍去球谐函数部分)

            第三小问更像是数学题,存在一个定值$a$使得$V(a) = < V >$,$r<a$时显然成立,接下来求这个积分的上限(这个积分值也是关于$V(r)$单增的),所以考虑积分到无穷处

            $$ \int_{0}^{+\infty} R(r)^{2}r^{2}dr = 1 $$
        
        \subsection{磁场下力学量的对易关系}
            $$ \comm{\vu*{p}_{i}}{\vu*{p}_{j}} = 0 $$

            同时对易不具有传递性:$ \vu*{A} $与$ \vu*{B} $对易,$ \vu*{B} $与$ \vu*{C} $对易,但是$ \vu*{A} $与$ \vu*{C} $不一定对易

            此题简单但是容易计算错误,建议再算几遍练手

        \subsection{维里定理一般形式的应用}
            维里定理告诉我们,知道势场就知道动能的平均值(和某些平均值相关),且隐含条件动能的平均值时显然大于0的

        \subsection{已知能量求解势能和作用程}
            求解到$ \cot{\alpha a} = - \dfrac{\beta}{\alpha} $,很显然有$ \alpha^{2} + \beta^{2} = \dfrac{2\mu V_{0}}{\hbar^{2}}  $
            
            所以需要这个关系消去不包含能量的$\beta$,$ \quad \sin^{2}{x} = \dfrac{1}{1 + \cot^{2}{x}} $

            算到最后一步的时候,$ \dfrac{E+V_{0}}{V_{0}} = 1 - \dfrac{\abs{E}}{V_{0}} $(计算的时候的能量仍旧是负值,结合能是正数)

        \subsection{连续势求解束缚定态能量}
            包含合流超几何方程,跳过
            
            但是具有启发性,当势函数是复杂的连续函数时,可以进行换元转化成可以求解的方程形式
            
        \subsection{中心力场转化为氢原子定态方程}
        $$ V_{r} = \frac{A}{r^{2}} - \frac{B}{r} $$
        如上形式的中心力场可以转化为氢原子的定态方程(有一项为$\frac{l(l+1)}{r^{2}}$可以合并半径的平方项)

        得到新的$l'$与$e^{2}$后需要把原来解的主量子数$ n = n_{r} + l + 1$中的$l$替换掉,也要替换掉玻尔半径的$ a = \frac{\hbar^{2}}{\mu e^{2}} $中的$e^{2}$

        再次回顾类氢原子的相关公式

        $$ V_{r} = - \dfrac{Ze^{2}}{r}  \quad  E_{n} = - \dfrac{Z^{2}e^{2} }{2an^{2}} $$




    \section{近似方法}

        \begin{formal}
            知识点总结  
            \begin{enumerate}
                \item 定态非简并微扰理论
                \begin{itemize}
                    \item 一级能量修正$ \pe{n}{1} = \ev{\vu*{H'}}{\pp{n}{0}} \quad $
                    一级波函数修正$ \pp{n}{1} = \sum\limits_{m \neq n} \dfrac{\mel{\pp{m}{0}}{\vu*{H'}}{\pp{n}{0}}}{\pe{n}{0} - \pe{m}{0} } \pp{m}{0} $
                    
                    \item 二级能量修正$ \pe{n}{2} = \sum\limits_{m \neq n} \dfrac{\abs{\mel{\pp{m}{0}}{\vu*{H'}}{\pp{n}{0}}}^{2}}{\pe{n}{0} - \pe{m}{0}} \quad $
                \end{itemize}

                \item 定态简并微扰理论
                
                体系的哈密顿量$\vu*{H} = \vu*{H}_{0} + \vu*{H}'$.已知$\vu*{H}_{0}$的本征能量与本征函数为$\pe{n}{0}$与$\pp{m}{0}$,$m=1,2,3,\cdots$,设$m=n$的某一定态能量$\pe{n}{0}$是
                k度简并的,与它相应的k个波函数记为$\varphi_{i},i=1,2,3,\cdots$.该定态的零级近似波函数为
                $$ \pp{n}{0} = \sum\limits_{i=1}^{k} c_{i} \varphi_{i} $$           
                
                系数${c_{i}}$满足方程
                $$ 
                \mqty(H_{11}'-\pe{n}{1} & H_{12}' & \cdots & H_{1k}' \\ H_{21}' & H_{12}'- \pe{n}{1} & \cdots & H_{2k}' \\ \vdots & \vdots & \ddots & \vdots \\ H_{k1}' & H_{k2}' & \cdots & H_{kk}'- \pe{n}{1} ) 
                \mqty(c_{1} \\ c_{2} \\ \vdots \\ c_{k}) = 0 
                $$

                $$ H_{ij}' = \int \varphi_{i}^{*} \vu*{H}' \varphi_{j} d\tau $$
                
                久期方程
                $$
                \mqty| H_{11}'-\pe{n}{1} & H_{12}' & \cdots & H_{1k}' \\ H_{21}' & H_{12}'- \pe{n}{1} & \cdots & H_{2k}' \\ \vdots & \vdots & \ddots & \vdots \\ H_{k1}' & H_{k2}' & \cdots & H_{kk}'- \pe{n}{1}| = 0
                $$

                解的一级修正能量$ \pe{n}{1} = \pe{n1}{1} , \pe{n2}{1} , \pe{nk}{1} $,将$ \pe{n}{1} = \pe{n\alpha}{1} (\alpha = 1,2,\cdots,k)$带入方程$c_{j}$满足的方程,求出系数${c_{i}(\alpha)}$,得到零级近似波函数
                $$ \pp{n}{0} = \sum\limits_{i=1}^{k}c_{i}(\alpha)\varphi_{i} \quad \alpha = 1,2,3.\cdots,k $$

                相应的一级近似能量为
                $$ E_{n\alpha} = \pe{n}{0} + \pe{n\alpha}{1} $$

                如果$\pe{n}{1}$有重根,即$E_{n\alpha}$中某个能态仍是简并的,则与该态相应的零级近似波函数不能确定,如果微扰矩阵$H'$是对角矩阵,则对角元素就是一级修正能量

                $$ \pe{i}{1} = H_{ii}' = \int \varphi_{i}^{*} \vu*{H} \varphi_{i} d\tau $$


                如果对角短阵的对角元素$H_{ii}'$取单—值(取值不同于其他对角元素),则相应的$\varphi_{i}$就是零级近似波函数.如果所有对角元素互不相等,则$\varphi_{1},\varphi_{2},\cdots,\varphi_{k}$都是零级近似波
                函数.这时简并态微扰问题可用非简并微扰论处理

                \item 已知粒子的哈密顿算符$\vu*{H}$及归一化试探波函数$\psi(\va{r},\alpha)$,其中$\alpha$为待定参数,计算
                $$ E_{\alpha} = \int \psi^{*}(\va{r},\alpha) \vu*{H} \psi(\va{r},\alpha) d\tau  $$

                由$ \pdv{E_{\alpha}}{\alpha} = 0 $,求出$\alpha$,代入$E_{\alpha}$得到基态的近似能量,再将$\alpha$代入试探波函数$\psi(\va{r},\alpha)$,得到基态的近似波函数

                \item 设$t<0$时粒子处于$\vu*{H}_{0}$的定态$\varphi_{k}$(能量为$E_{k}$),$t \geq 0$时粒子收到微扰$\vu*{H}'(t)$的作用,$t>0$时粒子跃迁到$\vu*{H_{0}}$的另一定态$\varphi_{m}$(能量为$E_{m}$)概率为
                $$ W_{k \rightarrow m}(t) = \dfrac{1}{\hbar^{2}} \abs{ \int_{0}^{t} H_{mk}'(t) e^{i\omega_{mk}t} dt }^{2} $$
                $$ H_{mk}'(t) = \int \varphi_{m}^{*}(\va{r}) \vu*{H}(t) \varphi_{k}(\va{r}) d\tau \qquad \omega_{mk} = \dfrac{E_{m}-E_{k}}{\hbar} $$

                \item 黄金规则公式
                $$ w = \frac{2\pi}{\hbar}  \abs{H_{mk}'(E)}^{2} \rho(E) $$

                这是在常微扰$\vu*{H}'$(不显含t)作用下体系由能量连续的$\psi_{k}$态到$\psi_{m}$态的跃迁速

                其中
                $$ H_{mk}'(E) = \int \psi_{m}^{*} \vu*{H}' \psi_{k} d\tau $$

                $\rho(E)$是能态密度——单位能量间隔的状态数

                \item 强度为$I(\omega)$的连续光照射原子发生由$\psi_{k}$态到$\psi_{m}$态的跃迁速率(电偶极近似)
                $$ w_{k \rightarrow m} = \frac{4\pi^{2}e^{2}}{3\hbar^{2}} I(\abs{\omega_{mk}}) \abs{r_{mk}}^{2} \quad \omega_{mk} = \dfrac{E_{m}-E_{k}}{\hbar} $$
                $$ \abs{r_{mk}}^{2} \equiv \abs{x_{mk}}^{2} + \abs{y_{mk}}^{2} + \abs{z_{mk}}^{2} $$
                $$ \abs{x_{mk}} = \int \psi_{m}^{*}x\psi_{k} d\tau  \quad \abs{y_{mk}} = \int \psi_{m}^{*}y\psi_{k} d\tau \quad \abs{z_{mk}} = \int \psi_{m}^{*}z\psi_{k} d\tau $$
                
                电偶极跃迁选择定则:$ \quad \triangle = \pm 1 \quad \triangle m = 0,\pm 1 $

                \item 原子的自发跃迁速率为
                $$ A_{k \rightarrow m} = \frac{4e^{2}\omega_{km}^{3}}{3\hbar c^{3}} \abs{r_{mk}}^{2} $$
    

            \end{enumerate}
        \end{formal}

    \section{自旋}

    \section{全同粒子体系}

    \section{散射}
    
    


  
\end{document}