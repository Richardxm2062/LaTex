% !TeX TS-program = xelatex

\documentclass{resume}

\usepackage{graphicx}

\ResumeName{马祥芸}


% 如果想插入照片,请使用以下两个库。
% \usepackage{graphicx}
% \usepackage{tikz}

\begin{document}

\ResumeContacts{
  (+86)177-250-25618,%
  \ResumeUrl{WeChatID}{WeChatID:Richardxm2062},%
  \ResumeUrl{Mail}{Email:richardxm2062@outlook.com},%
}

% 如果想插入照片,请取消此代码的注释。
% 但是默认不推荐插入照片,因为这不是简历的重点。
% 如果默认的照片插入格式不能满足你的需求,你可以尝试调整照片的大小,或者使用其他的插入照片的方法。
% 不然,也可以先渲染 PDF 简历,然后用其他工具在 PDF 上叠加照片。
% \begin{tikzpicture}[remember picture, overlay]
%   \node [anchor=north east, inner sep=1cm]  at (current page.north east) 
%      {\includegraphics[width=2cm]{image.png}};
% \end{tikzpicture}

\ResumeTitle


\section{教育经历}

\ResumeItem
[重庆大学|本科生]
{重庆大学}
[\textnormal{理论物理,物理学院,理学学士,GPA: 3.25/4.0\small{(专业前10/31)(年级前25/106)}}| ]   [2016.09—2020.06]

\section{获奖}
\begin{itemize}
  \item 2017年:丙等奖学金;优秀共青团员;科技文化展一等奖(校级)
  \item 2018年:重庆大学物理学术竞赛一等奖(校级);西南地区大学生物理学术竞赛一等奖(国家级);中国大学生物理学术竞赛三等(国家级);卓越杯物理实验竞赛二等(国家级)
  \item 2019年:丙等奖学金
\end{itemize}

\section{技术学习}
\begin{itemize}
  \item \textbf{语言}: [C;Perl;Fortran](课程学习);Python;Html;LaTex;Mathematica(符号计算语言)
  \item \textbf{其他}: Creo三维建模;Comsol物理场仿真;3D打印
\end{itemize}

\section{在校经历}

\begin{itemize}
  \item 2016年:加入物理学院科协学生会;参与重庆大学知识竞赛团体三等奖
  \item 2017年:成为班长管理班级;加入由刘院长创立的智能自组装机器人实验室(半年);参加数学学院举办的数模竞赛;初步接触全国大学生物理学术竞赛
  \item 2018年:参加物理学术竞赛(校级->国家级);选拔面试下一届竞赛成员;协助建立新的竞赛实验室;参加3D打印社团(校一级社团)并管理3d造(赞助商)网站重大板块
  \item 2019年:参加卓越杯物理实验竞赛
  \item 2020年:毕设优秀评级(前10\%);加入低温物理实验室,毕业后成为科研助理,创建实验室html主页
\end{itemize}


\section{个人总结}

\begin{itemize}
  \item 本人在校成绩优异,自主学习能力强,积极参加班级管理、校园活动、竞赛。具有良好的沟通能力和团队合作精神。
\end{itemize}

\end{document}
