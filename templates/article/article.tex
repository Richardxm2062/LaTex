\documentclass{article}    

\usepackage[UTF8, heading = false, scheme = plain]{ctex}    %解决中文字体,不改变排版
\usepackage{geometry}   %调整页边距等
\usepackage{indentfirst}    %首行缩进
\usepackage{graphicx}   %插入图片:\includegraphics{myimage.png}
\usepackage{hyperref}   %超链接插入包:\href{url}{name}
\usepackage{multirow}   % 插入表格用到的宏包:\begin{tabular}{|c|c|}
\usepackage{xcolor}  %颜色包:\color{}
\usepackage{listings}   %插入代码等:\begin{lstlisting}[breaklines=true,backgroundcolor=\color{lightgray},title=]
\usepackage{verbatim}   %使用 comment 环境进行注释

\geometry{a4paper,left=5em,right=5em,bottom=5em,top=5em}    %设置为a4paper最好,点击pacakge geometry 查看文档
\setlength{\parindent}{2em}    %2em(注意不支持rem)代表每一段的首行缩进两个字符,某一行不缩进时使用 \noindent
\hypersetup{hidelinks,colorlinks=true,linkcolor=black,urlcolor=blue}   %对hyperref 包进行预设
\newtheorem{thm}{定理}[section]     %定义一个新的环境 thm, 命名为定理,以 节 开始编号,数学论文中常用
\newtheorem*{proof}{证明}
\newtheorem{lemma}{引理}[section]
\newtheorem*{corollary}{推论}


\title{Title}
\author{Richardxm2062}


\begin{document} 
    \maketitle
    \tableofcontents
    \newpage 

    \section{Section1} 

    \paragraph{} This is paragraph.  

            \subparagraph{} This is subparagraph.

    \indent This is an indent.

    \subsection{Subsection2}  

    \subsubsection{This is subsubsection} 

    \textbf{Items}   
    
    \begin{itemize} %items without serial numbers.
        \item This is a item. 
        \item This is a item.
        \item This is a item.
    \end{itemize}

    \begin{enumerate}   %items with serial numbers.
        \item This is a item.
        \item This is a item.
        \item This is a item.
    \end{enumerate} 


    \textbf{Table} 


    \href{tablegenerator.com}{Latex表格生成网站}    %superlink


    \begin{tabular}{|c|c|} 
        \hline 
        a & b \\ 
        \hline 
        c & dcccc\\ 
        \hline 
    \end{tabular} 
\\
\\

    \textbf{Eaquations}

    \begin{itemize}
        \item inline mode:    % \textbf:bold text
        $a^2+b^2=c^2$ 

        \item display mode:   
        \[a^2+b^2=c^2\]        % \[ \]  equation without serial numbers
        \begin{equation}       % basic eaquation environment
            a^2+b^2=c^2
        \end{equation}

        
        for more complicate equation, plz use package \emph{amsmath}
    
    \end{itemize}


    \textbf{Therom}

    \begin{thm}
        这是一个定理
    \end{thm}
    

    \begin{lstlisting}[breaklines=true,backgroundcolor=\color{lightgray},title=Code]
        #includ<stido.h>
    \end{lstlisting}

    \begin{center}  % decide the position of the picture 
        \includegraphics[scale=0.2]{picture.jpeg}   % include picture and scale
    \end{center}

     

\end{document} 