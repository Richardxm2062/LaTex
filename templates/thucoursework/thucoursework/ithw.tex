% Homework template for Inference and Information
% UPDATE: September 26, 2017 by Xiangxiang
\documentclass[a4paper]{article}
\usepackage[UTF8]{ctex}
\ctexset{
proofname = \heiti{证明}
}
\usepackage{amsmath, amssymb, amsthm}
% amsmath: equation*, amssymb: mathbb, amsthm: proof
\usepackage{moreenum}
\usepackage{mathtools}
\usepackage{url}
\usepackage{bm}
\usepackage{enumitem}
\usepackage{graphicx}
\usepackage{subcaption}
\usepackage{booktabs} % toprule
\usepackage[mathcal]{eucal}
\usepackage[thehwcnt = 1]{iidef}

\thecourseinstitute{清华大学深圳研究生院}
\thecoursename{应用信息论}
\theterm{2018年春季学期}
\hwname{作业}
\slname{\heiti{解}}
\begin{document}
\courseheader
\name{YOUR NAME}

\begin{enumerate}
  \setlength{\itemsep}{3\parskip}

  \item 设$X$和$Y$是各有均值$m_x,m_y$,方差为$\sigma_x^2,\sigma_y^2$,且相互独立的高斯随机变量,已知$U=X+Y,V=X-Y$。试求$I(U;V)$。
\begin{solution}
$U,V$的联合分布是均值为$[\mu_x+\mu_y,\mu_x-\mu_y]$,协方差矩阵为
$$\Lambda_{U,V}=\begin{bmatrix}
1 & 1\\
1 & -1\\
\end{bmatrix}
\begin{bmatrix}
\sigma_x^2 & 0\\
0 & \sigma_x^2\\
\end{bmatrix}
\begin{bmatrix}
1 & 1\\
1 & -1\\
\end{bmatrix}^T
=\begin{bmatrix}
\sigma_x^2+\sigma_y^2 & \sigma_x^2-\sigma_y^2\\
\sigma_x^2-\sigma_y^2 & \sigma_x^2+\sigma_y^2\\
\end{bmatrix}
$$
由多元高斯分布微分熵的公式
$$
h(U)=\frac{1}{2}\log ((2\pi e)^2 |\Lambda_{U,V}|)=\frac{1}{2}\log(16\pi^2 e^2 \sigma^2_x\sigma^2_y)
$$
$U|V=v$也是高斯分布,方差为$\frac{4\sigma_x^2\sigma_y^2}{\sigma_x^2+\sigma_y^2}$,与$v$无关,因此
$$
h(U|V)=\E_{V}[h(U|V=v)]=\frac{1}{2}\log(2\pi e \frac{4\sigma_x^2\sigma_y^2}{\sigma_x^2+\sigma_y^2})\Rightarrow
$$
\begin{align*}
I(U;V)=& h(U)-h(U|V)\\
=&\frac{1}{2}\log(16\pi^2 e^2 \sigma^2_x\sigma^
2_y)-\frac{1}{2}\log(2\pi e \frac{4\sigma_x^2\sigma_y^2}{\sigma_x^2+\sigma_y^2})\\
=&\frac{1}{2}\log(2\pi e (\sigma_x^2+\sigma_y^2))
\end{align*}

%$$
%p(u,v)=\frac{1}{4\pi \sigma_x\sigma_y}\exp(-\frac{(\frac{u+v}{2}-\mu_x)^2+(\frac{u+v}{2}-\mu_y)^2}{2})
%$$

\end{solution}
\item 设有随机变量$X,Y,Z$均取值于$\{0,1\}$,已知$I(X;Y)=0,I(X;Y|Z)=1$。求证$H(Z)=1,H(X,Y,Z)=2$
\begin{proof}
$I(X;Y|Z)=H(X|Z)-H(X|Y,Z)\leq H(X|Z)\leq H(X)\leq \log(2)=1$
所以等号全都成立$\Rightarrow X\sim B(\frac{1}{2})$。
同理可知$Y\sim B(\frac{1}{2})$。
另外$H(Y|Z)=H(Y)\Rightarrow I(Y;Z)=0\Rightarrow H(Z|Y)=H(Z)$
\begin{align*}
&H(X|Y,Z)=0 \\
\iff &H(X,Y,Z)=H(Y,Z)\\
\iff &H(X,Y)+H(Z|X,Y)=H(Y)+H(Z|Y)\\
\iff &2+H(Z|X,Y)=1+H(Z)\\
\iff &H(Z)=1+H(Z|X,Y)
\end{align*}
由上式推出$H(Z)\geq 1$,又$H(Z)\leq 1\Rightarrow H(Z)=1\Rightarrow H(X,Y,Z)=2$
\end{proof}
\item
设有信号$X$经过处理器$A$后获输出$Y$,$Y$再经处理器$B$后获输出$Z$。已知处理器$A$和$B$
分别独立处理$X$和$Y$。试证:$I(X;Z)\leq I(X;Y)$
\begin{proof}
$I(X;Z)=H(Z)-H(Z|X)=H(Z);I(Y;Z)=H(Y)$因为$Z$是$Y$的函数$\Rightarrow H(Z)\leq H(Y) \Rightarrow I(X;Z)\leq I(X;Y)$
\end{proof}
\item 已知随机变量$X$和$Y$的联合概率密度$p(a_k,b_j)$满足
$$
p(a_1)=\frac{1}{2},p(a_2)=p(a_3)=\frac{1}{4},p(b_1)=\frac{2}{3},p(b_2)=p(b_3)=\frac{1}{6}
$$
试求能使$H(X,Y)$取得最大值的联合概率密度分布。
\begin{solution}
$H(X,Y)=H(X)+H(Y)-I(X;Y)\leq H(X)+H(Y)=\frac{7}{6}+\log 3$
等号成立当且仅当$X,Y$相互独立$\Rightarrow p(x,y)=p(x)p(y)$
\end{solution}
\item 设随机变量$X,Y,Z$满足$p(x,y,z)=p(x)p(y|x)p(z|y)$。求证$I(X;Y)\geq I(X;Y|Z)$
\begin{proof}
因为$p(x,y,z)=p(x)p(y|x)p(z|y,x)\Rightarrow p(z|y,x)=p(z|x)\Rightarrow x$ 与$z$关于$y$条件独立$\Rightarrow I(X;Y|Z)=H(X|Z)-H(X|Y,Z)= H(X|Z)-H(X|Y)\leq H(X)-H(X|Y) =I(X;Y)$
\end{proof}
\item 求证$I(X;Y;Z)=H(X,Y,Z)-H(X)-H(Y)-H(Z)+I(X;Y)+I(Y;Z)+I(Z;X)$,其中
$I(X;Y;Z)\triangleq I(X;Y)-I(X;Y|Z)$
\begin{proof}
\begin{align*}
I(X;Y;Z) =& I(X;Y)-I(X;Y|Z) \\
=& H(X)+H(Y)-H(X,Y)-(H(X|Z)-H(X|Y,Z))\\
=& H(X)+H(Y)-H(X,Y)-(H(X,Z)-H(Z))+H(X,Y,Z)-H(Y,Z)\\
=& H(X,Y,Z)-H(X)-H(Y)-H(Z)+(H(X)+H(Y)-H(X,Y))\\
+&(H(Y)+H(Z)-H(Y,Z))+(H(Z)+H(X)-H(X,Z))\\
=& H(X,Y,Z)-H(X)-H(Y)-H(Z)+I(X;Y)+I(Y;Z)+I(Z;X)
\end{align*}
\end{proof}
\item 令$p=(p_1,p_2,\dots,p_a)$是一个概率分布,满足$p_1\geq p_2\geq \dots p_a$,假设$\epsilon >0 $使得$p_1-\epsilon \geq p_2+\epsilon$成立,证明:$H(p_1,p_2,\dots,p_a)
\leq H(p_1-\epsilon,p_2+\epsilon,p_3,\dots,p_a)$
\begin{proof}
设$f(\epsilon)=(p_1-\epsilon)\log(p_1-\epsilon)+(p_2+\epsilon)\log(p_2+\epsilon)$
由已知$0\leq \epsilon \frac{p_2-p_1}{2}$
$f'(\epsilon)=\log\frac{p_2+\epsilon}{p_1-\epsilon}\leq 0$
$\Rightarrow f(\epsilon)\leq f(0)\Rightarrow H(p_1,p_2,\dots,p_a)\leq H(p_1-\epsilon,p_2+\epsilon,p_3,\dots,p_a)$
\end{proof}
\item 设$p_i(x)\sim N(\mu_i,\sigma_i^2)$,试求相对熵$D(p_2||p_1)$
\begin{solution}
\begin{align*}
D(p_2||p_1)=& \int_{\mathbb{R}} p_2(x) \log \frac{p_2(x)}{p_1(x)}dx\\
=& \int_{\mathbb{R}} p_2(x) \left(\log \frac{\sigma_1^2}{\sigma_2^2}+\frac{1}{2}((x-\mu_1)^2-(x-\mu_2)^2)\log e\right)dx\\
=& 2\log \frac{\sigma_1}{\sigma_2}+\frac{1}{2}(\mu_1^2-\mu_2^2)\log e+(\mu_2-\mu_1)\mu_2\log e\\
=& 2\log \frac{\sigma_1}{\sigma_2}+\frac{1}{2}(\mu_1-\mu_2)^2\log e
\end{align*}
\end{solution}
\item 若$f(x)$分别是区间$(0,0.01),(0,0.5),(0,1),(0,2),(0,5)$上均匀分布的分布函数,计算$f(x)$的微分熵。
\begin{solution}
设$U_t$是$(0,t)$上的均匀分布,则$h(U_t)=\log t$
\begin{itemize}
\item $h(U_{0.01})=\log 0.01$
\item $h(U_{0.5})=-1$
\item $h(U_{1})=0$
\item $h(U_{2})=1$
\item $h(U_{5})=\log 5$
\end{itemize}
\end{solution}
\item 设
\begin{align*}
p_1(x,y)=& \frac{1}{2\pi \sigma_x\sigma_y}\exp[-\frac{1}{2}(\frac{x^2}{\sigma_x^2}+\frac{y^2}{\sigma_y^2})]\\
p_2(x,y)=& \frac{1}{2\pi \sigma_x\sigma_y\sqrt{1-\rho^2}}\exp[-\frac{1}{2(1-\rho^2)}(\frac{x^2}{\sigma_x^2}-2\rho\frac{xy}{\sigma_x\sigma_y}+\frac{y^2}{\sigma_y^2})]
\end{align*}
试求$D(p_2||p_1)$和$I(X;Y)$,其中$X,Y\sim p_2$
\begin{solution}
\begin{align*}
D(p_2||p_1) = & \iint_{\mathbb{R}^2} p_2(x,y)\log \frac{p_2(x,y)}{p_1(x,y)}dxdy \\
 -&\frac{1}{2}\log(1-\rho^2)\\
-&\frac{1}{2}(\log e)\iint_{\mathbb{R}^2} p_2(x,y)\left[\frac{\rho^2 x^2}{\sigma_x^2(1-\rho^2)}+\frac{\rho^2 y^2}{\sigma_y^2(1-\rho^2)}-\frac{2\rho xy}{(1-\rho^2)\sigma_x\sigma_y}\right]dxdy\\
=&-\frac{1}{2}\log(1-\rho^2)
\end{align*}
$X|Y=y$服从高斯分布,方差为$(1-\rho^2)\sigma_x^2$
\begin{align*}
I(X;Y) = & h(X)-h(X|Y)\\
= & \frac{1}{2}\log(2\pi e \sigma_x^2) - \frac{1}{2}\log(2\pi e \sigma_x^2(1-\rho^2))\\
= & \frac{1}{2}\log(\frac{2\pi e}{1-\rho^2})
\end{align*}

\end{solution}

\end{enumerate}
\end{document}
\begin{equation}
\end{equation}

%%% Local Variables:
%%% mode: late\rvx
%%% TeX-master: t
%%% End:
