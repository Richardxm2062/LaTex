\documentclass{article}
\documentclass{article}    

%文章相关
\usepackage[UTF8, heading = false, scheme = plain]{ctex}    %解决中文字体,不改变排版
\usepackage{geometry}                                       %调整页边距等
\usepackage{indentfirst}                                    %首行缩进
\usepackage[dvipsnames,svgnames]{xcolor}                                         %颜色包:\color{}


%图片宏包
\usepackage{graphicx}                                       %插入图片:\includegraphics{myimage.png}
\usepackage{float}
\usepackage{caption}
\usepackage{subcaption}

%数学相关
\usepackage{amsmath,amsthm}                                 %amsmath包应该在前
\usepackage{braket}                                         %狄拉克符号系统 \bra{} \ket{}

%实用的内容说明包
\usepackage{hyperref}                                       %超链接插入包:\href{url}{name}
\usepackage{multirow}                                       %插入表格用到的宏包:\begin{tabular}{|c|c|}
\usepackage{listings}                                       %插入代码等:\begin{lstlisting}[breaklines=true,backgroundcolor=\color{lightgray},title=]
\usepackage{verbatim}                                       %使用 comment 环境进行注释


%文本底纹实现
\usepackage{lipsum}                                         %该宏包是通过 \lipsum 命令生成一段本文,正式使用时不需要引用该宏包
\usepackage[strict]{changepage}                             %提供一个 adjustwidth 环境
\usepackage{framed}                                         %实现方框效果
\usepackage{newtxtext}
\usepackage{tcolorbox}                                      %文本底纹包,放在xcolor包后
                               
% environment derived from framed.sty: see leftbar environment definition
\definecolor{formalshade}{rgb}{0.95,0.95,1} % 文本框颜色
% ------------------******-------------------
% 注意行末需要把空格注释掉,不然画出来的方框会有空白竖线
\newenvironment{formal}{%
\def\FrameCommand{%
\hspace{0em}%
{\color{Green}\vrule width 0.3em}%
\colorbox{greenshade}%
}%
\MakeFramed{\advance\hsize-\width\FrameRestore}%
\noindent\hspace{-2em}% disable indenting first paragraph
\begin{adjustwidth}{}{2.5em}%
\vspace{0.11em}\vspace{0.1em}%
}
{%
\vspace{2pt}\end{adjustwidth}\endMakeFramed%
}
% ----

\definecolor{greenshade}{rgb}{0.90,0.99,0.91}               %绿色文本框,竖线颜色设为 Green
\definecolor{redshade}{rgb}{1.00,0.90,0.90}                 %红色文本框,竖线颜色设为 LightCoral
\definecolor{brownshade}{rgb}{0.99,0.97,0.93}               %莫兰迪棕色,竖线颜色设为 BurlyWood




%预设
\geometry{a4paper,left=5em,right=5em,bottom=5em,top=5em}    %设置为a4paper最好,点击pacakge geometry 查看文档
\setlength{\parindent}{2em}                                 %2em(注意不支持rem)代表每一段的首行缩进两个字符,某一行不缩进时使用 \noindent

\hypersetup{hidelinks,colorlinks=true,
linkcolor=black,urlcolor=blue}                              %对hyperref 包进行预设

\newtheorem{thm}{定理}[section]                              %定义一个新的环境 thm, 命名为定理,以 节 开始编号,数学论文中常用

\renewcommand{\proofname}{ \qquad \bf 证明}                  %更改proof为中文证明

\newenvironment{solution}{\proof[\indent \bf 解]}
{\renewcommand{\qedsymbol}{}\endproof}                      %提供解环境

\newtheorem{lemma}{引理}[section]

\newtheorem{corollary}{推论}

%一些def
\def\thmindent{\setlength{\parindent}{5em}}                  %\thmindent
\def\pfindent{\setlength{\parindent}{5.5em}}                 %\pfindent
\def\clindent{\setlength{\parindent}{4em}}                   %clindent
\def\sdr{Schr\"{o}dinger}                                    %薛定谔名字
\def\intff{\int_{-\infty}^{+\infty}}                           %积分为(-\infty,+\infty)的积分



\title{高中物理}
\author{马祥芸}


\begin{document}
    \maketitle
    \tableofcontents
    \newpage

    \section{匀变速直线运动问题}

    \subsection{中间时刻/平均速度}
    中间时刻速度$v_{\frac{t}{2}}$与平均速度$\overline{v}$是同一个值
    $$
    v_{\frac{t}{2}} = v_{0} + \frac{at}{2} = \frac{v_{0}}{2} +  (\frac{v_{0}}{2} + \frac{at}{2})   = \frac{v_{0}+v_{t}}{2} = \overline{v}   
    $$
    中间位置速度
    
    \begin{numcases}{}
        \label{1} 2 a\frac{x}{2} = v_{\frac{x}{2}}^{2} - v_{0}^{2}  \\
        \label{2} 2 a\frac{x}{2} = v_{t}^{2} - v_{\frac{x}{2}}^{2} 
    \end{numcases}
    由方程$(1) - (2)$ 得到 $ v_{\frac{x}{2}} = \sqrt{\frac{v_{0}^{2} + v_{t}^{2}}{2}} $
    
    \subsection{纸带加速度问题}
    纸带的特点是,每个打印点的时间间隔相同均为$T$,且$x_{n}$规定的是第$n$个时间间隔内的位移,并非到起点的距离

    \begin{corollary*}
        相邻位移之间的差为$aT^{2}$,等时位移比例式为$x_{1}:x_{2}:x_{3} : \dots : x_{n} = 1:3:5: \dots :2n-1  $
    \end{corollary*}
    \begin{proof}
        \begin{align*}
            x_{n} = \frac{1}{2}a (nT)^{2} -  \frac{1}{2}a [(n-1)T]^{2} &= aT^{2} (\frac{2n-1}{2}) \\
            x_{n-1} &= aT^{2} (\frac{2n-3}{2})      \\
            x_{n} - x_{n-1} &= aT^{2}
        \end{align*}
    \end{proof}
    
    \begin{corollary*}
        等位移比例式子($1m$,$2m$,$3m \dots$)    \\
        前$1m,2m,3m \dots n \, m$所用时间比为$1:\sqrt{2}:\sqrt{3}:\dots:\sqrt{n}$,若是第$i\,m$内则向前减一个就行
        \begin{proof}
            \begin{align*}
                1 &= \frac{1}{2}a t_{1}^{2} \lra t_{1} =\sqrt{\frac{2}{a}} \vdot \sqrt{1}     \\
                2 &= \frac{1}{2}a t_{2}^{2} \lra t_{2} =\sqrt{\frac{2}{a}} \vdot \sqrt{2}     \\
                3 &= \frac{1}{2}a t_{3}^{2} \lra t_{3} =\sqrt{\frac{2}{a}} \vdot \sqrt{3}     \\  
                n &= \frac{1}{2}a t_{n}^{2} \lra t_{n} =\sqrt{\frac{2}{a}} \vdot \sqrt{n}     \\
            \end{align*}
            
            
            
        \end{proof}
    \end{corollary*}
    

    


    



\end{document}